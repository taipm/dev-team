% IMO 2023 Bài 1 - Lời giải Math Team
% Tạo bởi Math Team IMO Mode v2.0

\documentclass[11pt,a4paper]{article}

% XeLaTeX Vietnamese support
\usepackage{fontspec}
\usepackage[vietnamese]{babel}
\setmainfont{Times New Roman}

% Math packages
\usepackage{amsmath,amssymb,amsthm}
\usepackage{mathtools}

% Layout
\usepackage{geometry}
\geometry{margin=2.5cm}
\usepackage{xcolor}
\usepackage{tcolorbox}
\usepackage{array}
\usepackage{booktabs}

% Colors
\definecolor{trickcolor}{RGB}{0, 100, 180}
\definecolor{notecolor}{RGB}{180, 100, 0}
\definecolor{proofcolor}{RGB}{0, 120, 60}

% Custom boxes
\newtcolorbox{problembox}{
    colback=gray!10,
    colframe=black,
    title={\textbf{Bài 1}},
    fonttitle=\bfseries\large
}

\newtcolorbox{trickbox}{
    colback=trickcolor!10,
    colframe=trickcolor,
    title={\textbf{Ý tưởng chính}},
    fonttitle=\bfseries
}

\newtcolorbox{answerbox}{
    colback=proofcolor!10,
    colframe=proofcolor,
    fonttitle=\bfseries
}

% Header
\usepackage{fancyhdr}
\pagestyle{fancy}
\fancyhf{}
\rhead{{\small Độ khó: IMO P1}}
\lhead{{\small Số học}}
\cfoot{\thepage}

% Theorems - Vietnamese
\newtheorem{theorem}{Định lý}
\newtheorem{lemma}[theorem]{Bổ đề}
\newtheorem{claim}[theorem]{Khẳng định}

\begin{document}

\begin{center}
    {\LARGE\bfseries Olympic Toán Quốc tế 2023 - Bài 1}\\[0.3em]
    {\large Kỳ thi IMO lần thứ 64}\\[0.2em]
    {\small Chiba, Nhật Bản | Tháng 7/2023}
\end{center}

\vspace{0.5em}
\hrule
\vspace{1em}

%% ĐỀ BÀI
\begin{problembox}
Tìm tất cả các số nguyên hợp số $n > 1$ thỏa mãn tính chất sau: nếu $d_1, d_2, \ldots, d_k$ là tất cả các ước dương của $n$ với
\[1 = d_1 < d_2 < \cdots < d_k = n,\]
thì $d_i$ chia hết $d_{i+1} + d_{i+2}$ với mọi $1 \leq i \leq k-2$.
\end{problembox}

\vspace{1em}

%% Ý TƯỞNG CHÍNH
\begin{trickbox}
\textbf{Nhận xét quan trọng:} Cấu trúc ước của lũy thừa nguyên tố $p^a$ tạo thành cấp số nhân $\{1, p, p^2, \ldots, p^a\}$, làm cho điều kiện chia hết tự động thỏa mãn:
\[d_i = p^{i-1} \mid p^i + p^{i+1} = p^i(1+p) = d_{i+1}(1+p)\]

Với các số có $\geq 2$ ước nguyên tố phân biệt, luôn tồn tại ``điểm chuyển tiếp'' mà tại đó điều kiện bị vi phạm.
\end{trickbox}

\vspace{1em}

%% ĐÁP ÁN
\begin{answerbox}
\textbf{Đáp án:} $\boxed{n = p^a \text{ với } p \text{ là số nguyên tố và } a \geq 2}$

Tất cả các hợp số thỏa mãn điều kiện chính là các \textbf{lũy thừa nguyên tố} với số mũ ít nhất bằng 2.
\end{answerbox}

\vspace{1.5em}

%% LỜI GIẢI
\section*{Lời giải}

\subsection*{Bước 1: Khảo sát các trường hợp nhỏ}

\begin{center}
\begin{tabular}{c|c|l|c}
\toprule
$n$ & Lũy thừa nguyên tố? & Các ước & Thỏa mãn? \\
\midrule
$4 = 2^2$ & Có & $1, 2, 4$ & $\checkmark$ \\
$6 = 2 \cdot 3$ & Không & $1, 2, 3, 6$ & $\times$ $(2 \nmid 9)$ \\
$8 = 2^3$ & Có & $1, 2, 4, 8$ & $\checkmark$ \\
$9 = 3^2$ & Có & $1, 3, 9$ & $\checkmark$ \\
$10 = 2 \cdot 5$ & Không & $1, 2, 5, 10$ & $\times$ $(2 \nmid 15)$ \\
$12 = 2^2 \cdot 3$ & Không & $1, 2, 3, 4, 6, 12$ & $\times$ $(2 \nmid 7)$ \\
$16 = 2^4$ & Có & $1, 2, 4, 8, 16$ & $\checkmark$ \\
$25 = 5^2$ & Có & $1, 5, 25$ & $\checkmark$ \\
$27 = 3^3$ & Có & $1, 3, 9, 27$ & $\checkmark$ \\
\bottomrule
\end{tabular}
\end{center}

\textbf{Dự đoán:} Đáp án chính xác là các lũy thừa nguyên tố $n = p^a$ với $a \geq 2$.

\subsection*{Bước 2: Chứng minh lũy thừa nguyên tố thỏa mãn điều kiện}

\begin{theorem}
Với mọi số nguyên tố $p$ và số nguyên $a \geq 2$, số $n = p^a$ thỏa mãn tính chất đã cho.
\end{theorem}

\begin{proof}
Các ước của $n = p^a$ chính xác là $\{1, p, p^2, \ldots, p^a\}$, nên $d_j = p^{j-1}$ với $j = 1, 2, \ldots, a+1$.

Với mọi $1 \leq i \leq a-1$, ta kiểm tra $d_i \mid d_{i+1} + d_{i+2}$:
\begin{align*}
d_{i+1} + d_{i+2} &= p^i + p^{i+1} = p^i(1 + p)
\end{align*}

Vì $d_i = p^{i-1}$ chia hết $p^i = p \cdot p^{i-1}$, ta có:
\[d_i = p^{i-1} \mid p^i(1+p) = d_{i+1} + d_{i+2}\]

Do đó điều kiện đúng với mọi $i$ hợp lệ.
\end{proof}

\subsection*{Bước 3: Chứng minh các số không phải lũy thừa nguyên tố không thỏa mãn}

\begin{theorem}
Nếu $n$ có ít nhất hai ước nguyên tố phân biệt, thì $n$ không thỏa mãn điều kiện.
\end{theorem}

\begin{proof}
Gọi $p < q$ là hai ước nguyên tố nhỏ nhất phân biệt của $n$. Dãy ước bắt đầu bằng:
\[d_1 = 1, \quad d_2 = p\]

Chìa khóa là phân tích điều gì xảy ra tại $d_3$ và $d_4$.

\textbf{Trường hợp 1:} $q < p^2$ (nên $d_3 = q$)

Khi đó $d_4 \geq \min(p^2, pq)$.

Nếu $d_4 = p^2$: Ta cần $p \mid d_3 + d_4 = q + p^2$, tức là $p \mid q$. Nhưng $q$ là số nguyên tố và $q > p$, nên $p \nmid q$. \textbf{Mâu thuẫn.}

Nếu $d_4 = pq$: Ta cần $p \mid q + pq = q(1+p)$. Vì $\gcd(p,q) = 1$ và $p \nmid (1+p)$, điều này không thể xảy ra. \textbf{Mâu thuẫn.}

\textbf{Trường hợp 2:} $p^2 \leq q$ (nên $d_3 = p^2$ nếu $p^2 \mid n$)

Cuối cùng, $q$ phải xuất hiện trong dãy ước. Gọi $d_j = p^m$ và $d_{j+1} = q$ với $m$ nào đó.

Tại $i = j-1$: $d_i = p^{m-1}$, và ta cần $p^{m-1} \mid p^m + q$.

Vì $p^{m-1} \mid p^m$, ta cần $p^{m-1} \mid q$. Nhưng $q$ là số nguyên tố và $q \neq p$. \textbf{Mâu thuẫn.}

Trong mọi trường hợp, các số không phải lũy thừa nguyên tố đều không thỏa mãn điều kiện.
\end{proof}

\subsection*{Kết luận}

Tập hợp đầy đủ các hợp số thỏa mãn tính chất là:
\[\boxed{n = p^a \text{ với } p \text{ là số nguyên tố và } a \geq 2}\]

Ví dụ: $4, 8, 9, 16, 25, 27, 32, 49, 64, 81, 121, 125, \ldots$

\vspace{2em}

%% BÌNH LUẬN
\section*{Bình luận}

\subsection*{Kỹ thuật sử dụng}
\begin{itemize}
    \item \textbf{Khảo sát trường hợp nhỏ} - Thử $n = 4, 6, 8, \ldots$ để nhận ra quy luật
    \item \textbf{Phân tích chuỗi chia hết} - Nghiên cứu cấu trúc $d_i \mid d_{i+1} + d_{i+2}$
    \item \textbf{Chứng minh phản chứng} - Chỉ ra các số không phải lũy thừa nguyên tố không thỏa mãn
    \item \textbf{Cấu trúc phân tích thừa số} - Ý tưởng chính về dãy ước hình học và không hình học
\end{itemize}

\subsection*{Tại sao đây là bài IMO P1}
\begin{itemize}
    \item Điểm khởi đầu dễ tiếp cận (trường hợp nhỏ)
    \item Đáp án gọn gàng (lũy thừa nguyên tố)
    \item Yêu cầu phân tích trường hợp cẩn thận nhưng không cần lý thuyết sâu
    \item Kiểm tra độ trưởng thành toán học hơn là kiến thức chuyên sâu
\end{itemize}

\subsection*{Mở rộng}

\textbf{Câu hỏi:} Với những $n$ nào thì $d_i \mid d_{i+1} + d_{i+2} + \cdots + d_{i+m}$ đúng với mọi $i$ hợp lệ?

\textbf{Dự đoán:} Đáp án vẫn là lũy thừa nguyên tố, vì cấu trúc hình học của $\{1, p, p^2, \ldots\}$ làm cho mọi tổng đều chia hết cho các số hạng trước đó.

\vfill
\begin{center}
    {\small Math Team IMO Mode v2.0 --- Kiến trúc Solver Song song}\\
    {\footnotesize Ngày tạo: \today}
\end{center}

\end{document}
