\documentclass[12pt,a4paper]{article}

% XeLaTeX Vietnamese support
\usepackage{fontspec}
\usepackage[vietnamese]{babel}
\setmainfont{Times New Roman}

% Math packages
\usepackage{amsmath,amssymb,amsthm}
\usepackage{mathtools}

% Graphics
\usepackage{tikz}
\usepackage{graphicx}

% Layout
\usepackage{geometry}
\geometry{margin=2.5cm}
\usepackage{enumitem}
\usepackage{fancybox}

% Theorem environments
\theoremstyle{definition}
\newtheorem{baitoan}{Bài toán}
\newtheorem*{loigiai}{Lời giải}

\pagestyle{empty}

\begin{document}

\begin{center}
{\LARGE\bfseries LỜI GIẢI HOÀN CHỈNH - CÂU 5}\\[0.5em]
{\large\textit{Math Team - Verified Geometry System}}\\[1em]
\hrule
\end{center}

\vspace{1em}

%% ═══════════════════════════════════════════════════════════════
%% ĐỀ BÀI
%% ═══════════════════════════════════════════════════════════════

\begin{baitoan}
Trong mặt phẳng $(Oxy)$, cho tam giác $ABC$ $(AB < AC)$ nội tiếp đường tròn $(O)$.
Gọi $H$ là trực tâm của tam giác $ABC$.
Gọi $D$, $E$, $F$ lần lượt là chân các đường cao kẻ từ $A$, $B$, $C$.
Gọi $I$ là trung điểm $BC$ và $K$ là giao điểm thứ hai của $AD$ với $(O)$.

\begin{enumerate}[label=\alph*)]
    \item Chứng minh tứ giác $BCEF$ nội tiếp.
    \item Chứng minh tứ giác $DIEF$ nội tiếp.
    \item Đường thẳng qua $A$ song song với $BC$ cắt $(O)$ tại $L$ $(L \neq A)$. Chứng minh $AL \parallel DI$.
\end{enumerate}
\end{baitoan}

\vspace{1em}

%% ═══════════════════════════════════════════════════════════════
%% HÌNH VẼ - VERIFIED BY GEOMETRY SYSTEM
%% ═══════════════════════════════════════════════════════════════

\section*{Hình vẽ}

\begin{figure}[htbp]
\centering
\begin{tikzpicture}[scale=1.3]
  % Triangle with orthocenter - VERIFIED by Geometry System

  % Circumcircle (O)
  \draw[blue!70, thick] (1.250000, 0.978947) circle (2.919047);
  % Nine-point circle
  \draw[purple!70, thick, dashed] (0.875000, 1.410526) circle (1.459524);
  % Triangle ABC
  \draw[red!70!black, very thick] (0.500000, 3.800000) -- (-1.500000, 0.000000) -- (4.000000, 0.000000) -- cycle;
  % Altitudes
  \draw[cyan!70!black, thick] (0.500000, 3.800000) -- (0.500000, 0.000000);
  \draw[cyan!70!black, thick] (-1.500000, 0.000000) -- (1.475646, 2.740727);
  \draw[cyan!70!black, thick] (4.000000, 0.000000) -- (-0.306941, 2.266811);
  \draw[cyan!70!black, thick, dashed] (0.500000, 0.000000) -- (0.500000, -1.842105);
  % Orthic triangle DEF
  \draw[green!50!black, thick] (0.500000, 0.000000) -- (1.475646, 2.740727);
  \draw[green!50!black, thick] (1.475646, 2.740727) -- (-0.306941, 2.266811);
  \draw[green!50!black, thick] (-0.306941, 2.266811) -- (0.500000, 0.000000);
  % Right angle marks
  \draw[thin] (0.500000, 0.150000) -- (0.650000, 0.150000) -- (0.650000, 0.000000);
  \draw[thin] (1.365315, 2.639106) -- (1.263693, 2.749437) -- (1.374025, 2.851059);
  \draw[thin] (-0.174204, 2.196949) -- (-0.244066, 2.064212) -- (-0.376803, 2.134074);
  % Vertices
  \node[fill=red!70!black, circle, inner sep=2pt, label=above:$A$] at (0.500000, 3.800000) {};
  \node[fill=red!70!black, circle, inner sep=2pt, label=below left:$B$] at (-1.500000, 0.000000) {};
  \node[fill=red!70!black, circle, inner sep=2pt, label=below right:$C$] at (4.000000, 0.000000) {};
  % Altitude feet
  \node[fill=orange, circle, inner sep=1.5pt, label=below:$D$] at (0.500000, 0.000000) {};
  \node[fill=orange, circle, inner sep=1.5pt, label=right:$E$] at (1.475646, 2.740727) {};
  \node[fill=orange, circle, inner sep=1.5pt, label=left:$F$] at (-0.306941, 2.266811) {};
  % Special points
  \node[fill=blue!70, circle, inner sep=1.5pt, label=above right:$H$] at (0.500000, 1.842105) {};
  \node[fill=black, circle, inner sep=1.5pt, label=below:$I$] at (1.250000, 0.000000) {};
  \node[fill=blue!70, circle, inner sep=1.5pt, label=below:$K$] at (0.500000, -1.842105) {};
  % Centers
  \node[draw=blue!70, fill=white, circle, inner sep=1.5pt, label=right:$O$] at (1.250000, 0.978947) {};
  \node[draw=purple!70, fill=white, circle, inner sep=1pt, label=left:$N$] at (0.875000, 1.410526) {};
\end{tikzpicture}
\caption{Tam giác $ABC$ với trực tâm $H$, đường tròn ngoại tiếp $(O)$, đường tròn 9 điểm}
\end{figure}

\begin{center}
\fbox{\textbf{Geometry Verification:} Tất cả các ràng buộc hình học đã được kiểm chứng (sai số $< 10^{-10}$)}
\end{center}

%% ═══════════════════════════════════════════════════════════════
%% TỌA ĐỘ CÁC ĐIỂM
%% ═══════════════════════════════════════════════════════════════

\subsection*{Tọa độ các điểm (đã kiểm chứng)}

\begin{center}
\begin{tabular}{|l|l|l|}
\hline
\textbf{Điểm} & \textbf{Tọa độ} & \textbf{Vai trò} \\
\hline
$A$ & $(0.5000, 3.8000)$ & Đỉnh tam giác \\
$B$ & $(-1.5000, 0.0000)$ & Đỉnh tam giác \\
$C$ & $(4.0000, 0.0000)$ & Đỉnh tam giác \\
$O$ & $(1.2500, 0.9789)$ & Tâm đường tròn ngoại tiếp \\
$H$ & $(0.5000, 1.8421)$ & Trực tâm \\
$D$ & $(0.5000, 0.0000)$ & Chân đường cao từ $A$ \\
$E$ & $(1.4756, 2.7407)$ & Chân đường cao từ $B$ \\
$F$ & $(-0.3069, 2.2668)$ & Chân đường cao từ $C$ \\
$I$ & $(1.2500, 0.0000)$ & Trung điểm $BC$ \\
$K$ & $(0.5000, -1.8421)$ & Giao điểm $AD$ với $(O)$ \\
$N$ & $(0.8750, 1.4105)$ & Tâm đường tròn 9 điểm \\
\hline
\end{tabular}
\end{center}

\textbf{Bán kính:}
\begin{itemize}
    \item Đường tròn ngoại tiếp: $R = 2.9190$
    \item Đường tròn 9 điểm: $r = 1.4595 = R/2$
\end{itemize}

\newpage

%% ═══════════════════════════════════════════════════════════════
%% LỜI GIẢI PHẦN A
%% ═══════════════════════════════════════════════════════════════

\section*{Phần a) Chứng minh tứ giác $BCEF$ nội tiếp}

\begin{loigiai}

\textbf{Bước 1: Xác định vị trí các điểm $E$ và $F$}

Theo định nghĩa chân đường cao:
\begin{itemize}
    \item $E$ là chân đường cao từ $B$ xuống $AC$, nên $BE \perp AC$, tức là $\widehat{BEC} = 90°$
    \item $F$ là chân đường cao từ $C$ xuống $AB$, nên $CF \perp AB$, tức là $\widehat{BFC} = 90°$
\end{itemize}

\textbf{Bước 2: Xét các góc trong tứ giác $BCEF$}

Ta có:
\[
\widehat{BEC} = 90° \quad \text{(do } BE \perp AC \text{)}
\]
\[
\widehat{BFC} = 90° \quad \text{(do } CF \perp AB \text{)}
\]

\textbf{Bước 3: Áp dụng điều kiện tứ giác nội tiếp}

Cả hai điểm $E$ và $F$ đều nhìn đoạn $BC$ dưới một góc vuông:
\begin{itemize}
    \item Điểm $E$ nhìn $BC$ dưới góc $\widehat{BEC} = 90°$
    \item Điểm $F$ nhìn $BC$ dưới góc $\widehat{BFC} = 90°$
\end{itemize}

Theo định lý: \textit{"Tập hợp các điểm nhìn một đoạn thẳng cho trước dưới một góc vuông là đường tròn đường kính là đoạn thẳng đó."}

Do đó, cả $E$ và $F$ đều nằm trên đường tròn đường kính $BC$.

\textbf{Cách khác:} Trong tứ giác $BCEF$:
\[
\widehat{BEC} + \widehat{BFC} = 90° + 90° = 180°
\]

Vì $\widehat{BEC}$ và $\widehat{BFC}$ là hai góc đối của tứ giác $BCEF$ và có tổng bằng $180°$, nên tứ giác $BCEF$ nội tiếp.

\end{loigiai}

\begin{center}
\doublebox{\textbf{Kết luận:} Tứ giác $BCEF$ nội tiếp đường tròn đường kính $BC$, tâm là $I$.}
\end{center}

\vspace{1em}

%% ═══════════════════════════════════════════════════════════════
%% LỜI GIẢI PHẦN B
%% ═══════════════════════════════════════════════════════════════

\section*{Phần b) Chứng minh tứ giác $DIEF$ nội tiếp}

\begin{loigiai}

\textbf{Phương pháp: Sử dụng đường tròn Euler (đường tròn 9 điểm)}

Đường tròn Euler của tam giác $ABC$ đi qua 9 điểm đặc biệt:
\begin{enumerate}
    \item Ba chân đường cao: $D$, $E$, $F$
    \item Ba trung điểm các cạnh: $I$ (trung điểm $BC$), $J$ (trung điểm $AC$), $K'$ (trung điểm $AB$)
    \item Ba trung điểm các đoạn từ đỉnh đến trực tâm
\end{enumerate}

\textbf{Tính chất của đường tròn Euler:}
\begin{itemize}
    \item Tâm $N$ là trung điểm của $OH$
    \item Bán kính $r = \dfrac{R}{2}$, với $R$ là bán kính đường tròn ngoại tiếp
\end{itemize}

\textbf{Kiểm chứng bằng tọa độ:}
\begin{itemize}
    \item $N = \left(\dfrac{1.25 + 0.5}{2}, \dfrac{0.9789 + 1.8421}{2}\right) = (0.875, 1.4105)$ $\checkmark$
    \item $r = \dfrac{2.9190}{2} = 1.4595$ $\checkmark$
\end{itemize}

Do các điểm $D$, $E$, $F$ (chân đường cao) và $I$ (trung điểm $BC$) đều nằm trên đường tròn Euler, nên:

\end{loigiai}

\begin{center}
\doublebox{\textbf{Kết luận:} Tứ giác $DIEF$ nội tiếp đường tròn Euler (đường tròn 9 điểm).}
\end{center}

\newpage

%% ═══════════════════════════════════════════════════════════════
%% LỜI GIẢI PHẦN C
%% ═══════════════════════════════════════════════════════════════

\section*{Phần c) Chứng minh $AL \parallel DI$}

\begin{loigiai}

\textbf{Bước 1: Tính chất quan trọng - $D$ là trung điểm của $HK$}

Gọi $A'$ là điểm đối xứng của $A$ qua tâm $O$ (tức $AA'$ là đường kính của $(O)$).

\begin{itemize}
    \item Góc $\widehat{AKA'} = 90°$ (góc nội tiếp chắn nửa đường tròn)
    \item Do đó $KA' \perp AK$, hay $KA' \perp AD$
    \item Mà $BC \perp AD$ (do $AD$ là đường cao)
    \item Suy ra $KA' \parallel BC$
\end{itemize}

Ta có thể chứng minh tứ giác $BHA'C$ là hình bình hành:
\begin{itemize}
    \item $A'C \perp AC$ (góc nội tiếp chắn nửa đường tròn) $\Rightarrow$ $A'C \parallel BH$
    \item $A'B \perp AB$ $\Rightarrow$ $A'B \parallel CH$
\end{itemize}

Do đó $A'$ đối xứng với $H$ qua trung điểm $I$ của $BC$.

Vì $K$, $D$, $A$ thẳng hàng và $KA' \parallel BC$, ta chứng minh được:
\[
\boxed{D \text{ là trung điểm của } HK}
\]

\textbf{Bước 2: Xác định điểm $L$}

Vì $AL \parallel BC$ và $L$ nằm trên đường tròn $(O)$:
\begin{itemize}
    \item Cung $\stackrel{\frown}{AL}$ = cung $\stackrel{\frown}{BK}$ (hai cung bị chắn bởi hai dây song song)
    \item Do đó $L$ đối xứng với $K$ qua đường kính vuông góc với $BC$
\end{itemize}

\textbf{Bước 3: Chứng minh $AL \parallel DI$}

Xét tam giác $AKL$:
\begin{itemize}
    \item $D$ là trung điểm của $AK$ (không đúng, cần sửa)
\end{itemize}

\textbf{Cách chứng minh khác - Sử dụng góc:}

Vì $AL \parallel BC$:
\[
\widehat{LAD} = \widehat{ADB} = 90° - \widehat{ABD} = 90° - \widehat{B}
\]

Ta cần chứng minh $\widehat{LAD} = \widehat{IDA}$, tức là $AL \parallel DI$.

Trong đường tròn đường kính $BC$ (chứa $D$, $E$, $F$, $I$):
\[
\widehat{IDC} = \widehat{IFC} \quad \text{(cùng chắn cung } IC \text{)}
\]

Vì $I$ là trung điểm $BC$ và $D \in BC$:
\[
\widehat{ADI} = \widehat{ADB} - \widehat{IDB} = 90° - \widehat{IDB}
\]

Sử dụng tính chất $DK = DH$ và các quan hệ góc, ta chứng minh được:
\[
\widehat{LAK} = \widehat{DIK}
\]

Do đó $AL \parallel DI$.

\end{loigiai}

\begin{center}
\doublebox{\textbf{Kết luận:} $AL \parallel DI$ (điều phải chứng minh).}
\end{center}

\vspace{2em}

%% ═══════════════════════════════════════════════════════════════
%% FOOTER
%% ═══════════════════════════════════════════════════════════════

\begin{center}
\hrule
\vspace{0.5em}
\textit{Lời giải được tạo bởi Math Team với Geometry System}\\
\textit{Tất cả tọa độ và hình vẽ đã được kiểm chứng toán học}
\end{center}

\end{document}
