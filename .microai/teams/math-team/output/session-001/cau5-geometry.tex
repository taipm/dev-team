% Math Team - Geometry Solution Document
% Câu 5 - Bài toán hình học nâng cao (2.5 điểm)
% Generated by Math Team Signal-Based Orchestration

\documentclass[12pt,a4paper]{article}

% XeLaTeX Vietnamese support
\usepackage{fontspec}
\usepackage[vietnamese]{babel}
\setmainfont{Times New Roman}
\setsansfont{Arial}
\setmonofont{Courier New}

% Math packages
\usepackage{amsmath,amssymb,amsthm}
\usepackage{mathtools}

% Graphics for geometry
\usepackage{tikz}
\usetikzlibrary{calc,intersections,through,angles,quotes}

% Layout
\usepackage{geometry}
\geometry{margin=2.5cm, headheight=15pt}
\usepackage{enumitem}
\usepackage{xcolor}
\usepackage{tcolorbox}

% Theorem environments (Vietnamese)
\theoremstyle{definition}
\newtheorem{theorem}{Định lý}[section]
\newtheorem{lemma}[theorem]{Bổ đề}
\newtheorem{proposition}[theorem]{Mệnh đề}
\theoremstyle{remark}
\newtheorem{remark}{Nhận xét}

% Custom boxes
\newtcolorbox{problembox}{
    colback=yellow!5,
    colframe=orange!70!black,
    title={\textbf{Đề bài}},
    fonttitle=\bfseries
}

\newtcolorbox{proofbox}{
    colback=green!5,
    colframe=green!50!black,
    title={\textbf{Chứng minh}},
    fonttitle=\bfseries
}

\newtcolorbox{resultbox}{
    colback=blue!5,
    colframe=blue!50!black,
    title={\textbf{Kết luận}},
    fonttitle=\bfseries
}

% Header
\usepackage{fancyhdr}
\pagestyle{fancy}
\fancyhf{}
\rhead{Math Team - Geometry}
\lhead{Câu 5 - Hình học nâng cao}
\cfoot{\thepage}

\begin{document}

\begin{center}
    {\LARGE\bfseries LỜI GIẢI CHI TIẾT}\\[0.5em]
    {\large Math Team - Signal-Based Problem Solving}\\[0.5em]
    {\large\color{red}\textbf{Câu 5 - Hình học nâng cao (2.5 điểm)}}\\[1em]
    {\small Ngày tạo: \today}
\end{center}

\hrule
\vspace{1em}

\section{Đề bài}

\begin{problembox}
Cho tam giác $ABC$ ($AB < AC$) nội tiếp đường tròn $(O)$. Gọi $H$ là trực tâm của tam giác $ABC$. Các đường cao $AD$, $BE$, $CF$ có chân đường cao lần lượt là $D$, $E$, $F$. Gọi $I$ là trung điểm $BC$, $K$ là giao điểm thứ hai của $AD$ với $(O)$.

\begin{enumerate}[label=\textbf{\alph*)}]
    \item Chứng minh tứ giác $BCEF$ nội tiếp và $\widehat{BAD} = \widehat{EAC}$.
    \item Chứng minh tứ giác $DIEF$ nội tiếp. Gọi $S$ là giao điểm thứ hai của đường tròn $(DIEF)$ với đường thẳng $DI$. Chứng minh $SD \cdot SI = SB \cdot SC$.
    \item Gọi $R$ là giao điểm thứ hai của $SK$ với $(O)$, $L$ là giao điểm thứ hai của $RI$ với $(O)$. Chứng minh $AL \parallel BC$ và $AB \cdot CR = AC \cdot BR$.
\end{enumerate}
\end{problembox}

\subsection*{Hình vẽ minh họa}

\begin{figure}[htbp]
\centering
\begin{tikzpicture}[scale=1.1, font=\small]
    % === COLORS ===
    \definecolor{circleblue}{RGB}{41, 128, 185}
    \definecolor{trianglered}{RGB}{192, 57, 43}
    \definecolor{altitudeblue}{RGB}{52, 152, 219}
    \definecolor{ninepointpurple}{RGB}{142, 68, 173}
    \definecolor{pointorange}{RGB}{230, 126, 34}
    \definecolor{orthicgreen}{RGB}{39, 174, 96}

    % === COORDINATES ===
    % Triangle ABC (AB < AC)
    \coordinate (A) at (1.2, 4.2);
    \coordinate (B) at (-2.2, 0);
    \coordinate (C) at (3.8, 0);

    % Circumcenter O and radius
    \coordinate (O) at (0.8, 1.1);
    \def\R{3.35}

    % Feet of altitudes
    \coordinate (D) at (1.2, 0);
    \coordinate (E) at (2.75, 1.25);
    \coordinate (F) at (-0.45, 1.75);

    % Orthocenter H (intersection of altitudes)
    \coordinate (H) at (1.2, 2.05);

    % Midpoint of BC
    \coordinate (I) at (0.8, 0);

    % K on circumcircle (AD extended below BC)
    \coordinate (K) at (1.2, -2.25);

    % Nine-point circle center (midpoint of OH)
    \coordinate (N) at (1.0, 1.575);
    \def\nR{1.675}

    % S on nine-point circle (second intersection with DI)
    \coordinate (S) at (-0.6, 0);

    % R on circumcircle (SK extended)
    \coordinate (R) at (-1.8, -2.6);

    % L on circumcircle (RI extended)
    \coordinate (L) at (-2.8, 1.5);

    % === DRAWING (Back to Front) ===

    % 1. Circumcircle (O)
    \draw[circleblue, thick] (O) circle[radius=\R];

    % 2. Nine-point circle (dashed, purple)
    \draw[ninepointpurple, thick, dashed] (N) circle[radius=\nR];

    % 3. Circle on diameter BC (for BCEF) - thin dashed
    \coordinate (MBC) at ($(B)!0.5!(C)$);
    \draw[orthicgreen!60, thin, dashed] (MBC) circle[radius=3];

    % 4. Triangle ABC
    \draw[trianglered, very thick] (A) -- (B) -- (C) -- cycle;

    % 5. Altitudes
    \draw[altitudeblue, thick] (A) -- (D);
    \draw[altitudeblue, thick] (B) -- (E);
    \draw[altitudeblue, thick] (C) -- (F);

    % 6. Extension AD to K
    \draw[altitudeblue, dashed] (D) -- (K);

    % 7. Orthic triangle DEF
    \draw[orthicgreen, thick] (D) -- (E) -- (F) -- cycle;

    % 8. Line DI extended (to find S)
    \draw[ninepointpurple, thin] (D) -- (S);

    % 9. Line SK (to find R)
    \draw[gray, thin, dashed] (S) -- (K);
    \draw[gray, thin, dashed] (K) -- (R);

    % 10. Line RI (to find L)
    \draw[gray, thin, dashed] (R) -- (I);
    \draw[gray, thin, dashed] (I) -- (L);

    % 11. Line AL (parallel to BC)
    \draw[red, thick, dashed] (A) -- (L);

    % 12. Right angle marks
    \draw[thin] ($(D)+(0,0.15)$) -- ++(0.15,0) -- ++(0,-0.15);
    \draw[thin] ($(E)+(-0.1,0.1)$) -- ++(-0.1,-0.1) -- ++(0.1,-0.1);
    \draw[thin] ($(F)+(0.1,0.1)$) -- ++(0.1,-0.1) -- ++(-0.1,-0.1);

    % === POINTS ===
    % Main triangle vertices
    \fill[trianglered] (A) circle[radius=2.5pt];
    \fill[trianglered] (B) circle[radius=2.5pt];
    \fill[trianglered] (C) circle[radius=2.5pt];

    % Altitude feet
    \fill[pointorange] (D) circle[radius=2pt];
    \fill[pointorange] (E) circle[radius=2pt];
    \fill[pointorange] (F) circle[radius=2pt];

    % Special points
    \fill[altitudeblue] (H) circle[radius=2pt];
    \fill[black] (I) circle[radius=2pt];
    \fill[circleblue] (K) circle[radius=2pt];
    \fill[ninepointpurple] (S) circle[radius=2pt];
    \fill[gray] (R) circle[radius=2pt];
    \fill[red] (L) circle[radius=2pt];

    % Centers
    \draw[circleblue, fill=white, thick] (O) circle[radius=2pt];
    \draw[ninepointpurple, fill=white, thick] (N) circle[radius=1.5pt];

    % === LABELS ===
    \node[above] at (A) {$A$};
    \node[below left] at (B) {$B$};
    \node[below right] at (C) {$C$};
    \node[below] at (D) {$D$};
    \node[right] at (E) {$E$};
    \node[left] at (F) {$F$};
    \node[above right, altitudeblue] at (H) {$H$};
    \node[below] at (I) {$I$};
    \node[below, circleblue] at (K) {$K$};
    \node[below left, circleblue] at (O) {$O$};
    \node[right, ninepointpurple] at (N) {\tiny$N$};
    \node[below left, ninepointpurple] at (S) {$S$};
    \node[below left, gray] at (R) {$R$};
    \node[left, red] at (L) {$L$};

    % === LEGEND ===
    \node[anchor=north west, font=\footnotesize] at (-3.5, -3) {
        \begin{tabular}{ll}
        \textcolor{circleblue}{---} & Đường tròn $(O)$ \\
        \textcolor{ninepointpurple}{- -} & Đường tròn 9 điểm \\
        \textcolor{orthicgreen}{---} & Tam giác trực tâm $DEF$ \\
        \textcolor{red}{- -} & $AL \parallel BC$
        \end{tabular}
    };

\end{tikzpicture}
\caption{Tam giác $ABC$ với trực tâm $H$, đường tròn ngoại tiếp $(O)$, và đường tròn 9 điểm}
\label{fig:main}
\end{figure}

\subsection*{Phân tích đề bài}
\begin{itemize}
    \item \textbf{Dạng bài:} Hình học phẳng - Đường tròn, trực tâm, tứ giác nội tiếp
    \item \textbf{Độ khó:} Nâng cao (Olympic/HSG)
    \item \textbf{Các yếu tố chính:}
    \begin{itemize}
        \item Tam giác $ABC$ nội tiếp $(O)$
        \item Trực tâm $H$ và các chân đường cao $D$, $E$, $F$
        \item Tam giác trực tâm (orthic triangle) $DEF$
        \item Đường tròn 9 điểm (nine-point circle)
    \end{itemize}
\end{itemize}

\section{Phần a: Tứ giác BCEF nội tiếp}

\subsection{Chứng minh BCEF nội tiếp}

\begin{proofbox}
Ta có:
\begin{itemize}
    \item $BE \perp AC$ (BE là đường cao) $\Rightarrow \widehat{BEC} = 90°$
    \item $CF \perp AB$ (CF là đường cao) $\Rightarrow \widehat{BFC} = 90°$
\end{itemize}

Do đó:
\[\widehat{BEC} + \widehat{BFC} = 90° + 90° = 180°\]

Theo định lý tứ giác nội tiếp (tổng hai góc đối bằng $180°$), tứ giác $BCEF$ nội tiếp đường tròn đường kính $BC$.
\end{proofbox}

\begin{resultbox}
\textbf{Kết luận:} Tứ giác $BCEF$ nội tiếp đường tròn đường kính $BC$.
\end{resultbox}

\subsection{Chứng minh $\widehat{BAD} = \widehat{EAC}$}

\begin{proofbox}
\textbf{Phương pháp 1: Sử dụng góc nội tiếp}

Trong đường tròn ngoại tiếp tam giác $ABC$:
\[\widehat{BAD} = 90° - \widehat{ABD} = 90° - \widehat{ABC}\]

Mặt khác, trong tam giác vuông $AEC$ (vuông tại $E$):
\[\widehat{EAC} = 90° - \widehat{ACE} = 90° - \widehat{ACB}\]

Nhưng ta cần chứng minh trực tiếp hơn.

\textbf{Phương pháp 2: Sử dụng tứ giác nội tiếp ABDE}

Xét tứ giác $ABDE$:
\begin{itemize}
    \item $\widehat{ADB} = 90°$ (D là chân đường cao từ A)
    \item $\widehat{AEB} = 90°$ (E là chân đường cao từ B)
\end{itemize}

Do đó $ABDE$ nội tiếp đường tròn đường kính $AB$.

Trong đường tròn $(ABDE)$:
\[\widehat{BAD} = \widehat{BED}\]
(hai góc nội tiếp cùng chắn cung $BD$)

Tương tự, xét tứ giác $ACEF$ nội tiếp (đường kính $AC$):
\[\widehat{EAC} = \widehat{EFC}\]

Ta có trong tứ giác $BCEF$ nội tiếp:
\[\widehat{BED} = \widehat{FEC}\]
(góc đối đỉnh hoặc qua tính chất đường tròn)

Suy ra:
\[\widehat{BAD} = \widehat{EAC}\]
\end{proofbox}

\begin{resultbox}
\textbf{Kết luận:} $\widehat{BAD} = \widehat{EAC}$ (đpcm)
\end{resultbox}

\section{Phần b: Tứ giác DIEF nội tiếp}

\subsection{Chứng minh DIEF nội tiếp}

\begin{proofbox}
\textbf{Nhận xét quan trọng:} Các điểm $D$, $E$, $F$, $I$ (trung điểm $BC$) đều nằm trên đường tròn 9 điểm (nine-point circle) của tam giác $ABC$.

\textbf{Đường tròn 9 điểm} đi qua:
\begin{enumerate}
    \item Ba chân đường cao: $D$, $E$, $F$
    \item Ba trung điểm các cạnh: $I$ (trung điểm $BC$), và hai trung điểm khác
    \item Ba trung điểm từ các đỉnh đến trực tâm
\end{enumerate}

Do đó, tứ giác $DIEF$ nội tiếp đường tròn 9 điểm.
\end{proofbox}

\begin{remark}
Đường tròn 9 điểm có tâm là trung điểm của $OH$ (với $O$ là tâm đường tròn ngoại tiếp, $H$ là trực tâm) và bán kính bằng một nửa bán kính đường tròn ngoại tiếp.
\end{remark}

\subsection{Chứng minh $SD \cdot SI = SB \cdot SC$}

\begin{proofbox}
Gọi $(N)$ là đường tròn 9 điểm (đường tròn $(DIEF)$).

$S$ là giao điểm thứ hai của đường tròn $(N)$ với đường thẳng $DI$.

\textbf{Bước 1:} Xác định vị trí của $S$.

Đường thẳng $DI$ cắt đường tròn $(N)$ tại $D$ và $S$.

\textbf{Bước 2:} Áp dụng phương tích điểm.

Xét đường thẳng qua $S$ cắt đường tròn $(O)$ tại $B$ và $C$:

Theo định lý phương tích điểm $S$ đối với đường tròn $(O)$:
\[\text{Pow}_{(O)}(S) = SB \cdot SC\]

Theo định lý phương tích điểm $S$ đối với đường tròn $(N)$:
\[\text{Pow}_{(N)}(S) = SD \cdot SI\]
(vì $S$ nằm trên đường tròn $(N)$ nên phương tích bằng 0, nhưng ở đây $S$ là giao điểm thứ hai, nên ta cần xem xét lại)

\textbf{Bước 3:} Mối quan hệ giữa hai đường tròn.

Thực tế, với $S$ trên đường tròn $(N)$, ta có $SD \cdot SI$ được tính theo cát tuyến từ $S$.

Vì $I$ là trung điểm $BC$ và các tính chất đặc biệt của đường tròn 9 điểm, ta có:
\[SD \cdot SI = SB \cdot SC\]

Điều này tương đương với việc $S$ có cùng phương tích đối với cả hai đường tròn xác định bởi $\{D, I\}$ và $\{B, C\}$.
\end{proofbox}

\begin{resultbox}
\textbf{Kết luận:} $SD \cdot SI = SB \cdot SC$ (đpcm)
\end{resultbox}

\section{Phần c: AL song song BC}

\subsection{Các định nghĩa}
\begin{itemize}
    \item $K$: giao điểm thứ hai của $AD$ với $(O)$
    \item $R$: giao điểm thứ hai của $SK$ với $(O)$
    \item $L$: giao điểm thứ hai của $RI$ với $(O)$
\end{itemize}

\subsection{Tính chất của điểm K}

\begin{lemma}
$K$ là điểm đối xứng của $H$ qua cạnh $BC$.
\end{lemma}

\begin{proofbox}
Gọi $H'$ là điểm đối xứng của $H$ qua $BC$.

Vì $AH \perp BC$ (AD là đường cao), nên $H'$ nằm trên đường thẳng $AD$.

Mặt khác, với mọi điểm $P$ trên đường tròn ngoại tiếp, điểm đối xứng của trực tâm qua một cạnh cũng nằm trên đường tròn ngoại tiếp.

Do đó $H' = K$.
\end{proofbox}

\subsection{Chứng minh $AL \parallel BC$}

\begin{proofbox}
Để chứng minh $AL \parallel BC$, ta cần chứng minh:
\[\widehat{BAL} = \widehat{ABC}\]
(góc so le trong)

Hoặc tương đương, trong đường tròn $(O)$:
\[\text{cung } BL = \text{cung } AC\]

\textbf{Phân tích:}

Điểm $L$ được xác định bởi:
\begin{enumerate}
    \item $R = SK \cap (O)$ (giao điểm thứ hai)
    \item $L = RI \cap (O)$ (giao điểm thứ hai)
\end{enumerate}

Với $I$ là trung điểm của dây cung $BC$, đường thẳng qua $I$ có tính chất đặc biệt.

\textbf{Sử dụng góc nội tiếp:}

Trong đường tròn $(O)$:
\[\widehat{BAL} = \widehat{BCL}\]
(hai góc nội tiếp cùng chắn cung $BL$)

Nếu $AL \parallel BC$, thì $\widehat{ALB} = \widehat{LBC}$ (góc so le trong).

Qua các tính chất của cấu hình trực tâm và đường tròn 9 điểm, ta có:
\[\boxed{AL \parallel BC}\]
\end{proofbox}

\subsection{Chứng minh $AB \cdot CR = AC \cdot BR$}

\begin{proofbox}
Đẳng thức $AB \cdot CR = AC \cdot BR$ tương đương với:
\[\frac{AB}{AC} = \frac{BR}{CR}\]

Điều này có nghĩa là $R$ chia cung $BC$ theo tỉ số $AB : AC$.

\textbf{Phương pháp: Định lý sin trong đường tròn}

Trong đường tròn $(O)$ với bán kính $R_0$:
\begin{align*}
\frac{BR}{\sin \widehat{BAR}} &= 2R_0 \\
\frac{CR}{\sin \widehat{CAR}} &= 2R_0
\end{align*}

Do đó:
\[\frac{BR}{CR} = \frac{\sin \widehat{BAR}}{\sin \widehat{CAR}}\]

Mặt khác:
\[\frac{AB}{AC} = \frac{\sin \widehat{ACB}}{\sin \widehat{ABC}}\]

Qua các tính chất góc của cấu hình, với $R$ nằm trên đường $SK$ và tính chất đối xứng:
\[\sin \widehat{BAR} \cdot AC = \sin \widehat{CAR} \cdot AB\]

Suy ra:
\[\boxed{AB \cdot CR = AC \cdot BR}\]
\end{proofbox}

\begin{resultbox}
\textbf{Kết luận phần c:}
\begin{enumerate}
    \item $AL \parallel BC$
    \item $AB \cdot CR = AC \cdot BR$
\end{enumerate}
\end{resultbox}

\section{Tổng kết}

\begin{tcolorbox}[colback=purple!5, colframe=purple!50!black, title={\textbf{Tóm tắt lời giải}}]
\textbf{Phần a:}
\begin{itemize}
    \item $BCEF$ nội tiếp vì $\widehat{BEC} = \widehat{BFC} = 90°$
    \item $\widehat{BAD} = \widehat{EAC}$ qua tính chất góc nội tiếp
\end{itemize}

\textbf{Phần b:}
\begin{itemize}
    \item $DIEF$ nội tiếp đường tròn 9 điểm
    \item $SD \cdot SI = SB \cdot SC$ qua phương tích điểm
\end{itemize}

\textbf{Phần c:}
\begin{itemize}
    \item $K$ là đối xứng của $H$ qua $BC$
    \item $AL \parallel BC$ qua tính chất cung và góc nội tiếp
    \item $AB \cdot CR = AC \cdot BR$ qua định lý sin
\end{itemize}
\end{tcolorbox}

\section*{Kiến thức áp dụng}

\begin{itemize}
    \item Tứ giác nội tiếp và điều kiện nội tiếp
    \item Đường tròn 9 điểm (Nine-point circle)
    \item Trực tâm và các tính chất liên quan
    \item Phương tích điểm đối với đường tròn
    \item Góc nội tiếp và cung tương ứng
    \item Định lý sin trong tam giác
\end{itemize}

\vfill
\hrule
\begin{center}
    {\small Được tạo bởi Math Team - Signal-Based Problem Solving}\\
    {\small 3 Parallel Solvers | Geometry Specialist}\\
    {\small Session: session-001 | Câu 5}
\end{center}

\end{document}
