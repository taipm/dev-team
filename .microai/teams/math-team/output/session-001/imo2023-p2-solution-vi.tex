% IMO 2023 Bài 2 - Lời giải Math Team
% Tạo bởi Math Team IMO Mode v2.0

\documentclass[11pt,a4paper]{article}

% XeLaTeX Vietnamese support
\usepackage{fontspec}
\usepackage[vietnamese]{babel}
\setmainfont{Times New Roman}

% Math packages
\usepackage{amsmath,amssymb,amsthm}
\usepackage{mathtools}

% Graphics
\usepackage{tikz}
\usetikzlibrary{calc,intersections,through}

% Layout
\usepackage{geometry}
\geometry{margin=2.5cm}
\usepackage{xcolor}
\usepackage{tcolorbox}

% Colors
\definecolor{trickcolor}{RGB}{0, 100, 180}
\definecolor{proofcolor}{RGB}{0, 120, 60}

% Custom boxes
\newtcolorbox{problembox}{
    colback=gray!10,
    colframe=black,
    title={\textbf{Bài 2 (Hình học G4)}},
    fonttitle=\bfseries\large
}

\newtcolorbox{trickbox}{
    colback=trickcolor!10,
    colframe=trickcolor,
    title={\textbf{Ý tưởng chính}},
    fonttitle=\bfseries
}

\newtcolorbox{lemmabox}{
    colback=proofcolor!10,
    colframe=proofcolor,
    fonttitle=\bfseries
}

% Header
\usepackage{fancyhdr}
\pagestyle{fancy}
\fancyhf{}
\rhead{{\small Độ khó: IMO P2 (G4)}}
\lhead{{\small Hình học}}
\cfoot{\thepage}

% Theorems
\newtheorem{theorem}{Định lý}
\newtheorem{lemma}[theorem]{Bổ đề}
\newtheorem{claim}[theorem]{Khẳng định}

\begin{document}

\begin{center}
    {\LARGE\bfseries Olympic Toán Quốc tế 2023 - Bài 2}\\[0.3em]
    {\large Kỳ thi IMO lần thứ 64}\\[0.2em]
    {\small Chiba, Nhật Bản | Tháng 7/2023}
\end{center}

\vspace{0.5em}
\hrule
\vspace{1em}

%% ĐỀ BÀI
\begin{problembox}
Cho tam giác $ABC$ nhọn với $AB < AC$. Gọi $\Omega$ là đường tròn ngoại tiếp tam giác $ABC$. Gọi $S$ là điểm chính giữa cung $CB$ của $\Omega$ chứa $A$. Đường vuông góc từ $A$ đến $BC$ cắt $BS$ tại $D$ và cắt $\Omega$ lần nữa tại $E \neq A$. Đường thẳng qua $D$ song song với $BC$ cắt đường thẳng $BE$ tại $L$.

Gọi $\omega$ là đường tròn ngoại tiếp tam giác $BDL$. Cho $\omega$ cắt $\Omega$ lần nữa tại $P \neq B$.

\textbf{Chứng minh rằng} đường thẳng tiếp tuyến với $\omega$ tại $P$ cắt đường thẳng $BS$ trên đường phân giác trong của góc $\angle BAC$.
\end{problembox}

\vspace{1em}

%% Ý TƯỞNG CHÍNH
\begin{trickbox}
\textbf{Nhận xét quan trọng:}
\begin{itemize}
    \item $S$ là điểm chính giữa cung $CB$ chứa $A$ $\Rightarrow$ $S$ nằm trên đường phân giác trong của $\angle BAC$
    \item Bài toán thực chất yêu cầu chứng minh: \textbf{Tiếp tuyến của $\omega$ tại $P$ đi qua $S$}
    \item Để tiếp tuyến từ $S$ đến $\omega$ tiếp xúc tại $P$, cần: $SP^2 = \text{pow}_\omega(S)$
\end{itemize}
\end{trickbox}

\vspace{1em}

%% HÌNH VẼ
\begin{center}
\begin{tikzpicture}[scale=2.2]
    % Định nghĩa các điểm
    \coordinate (B) at (-1.2, 0);
    \coordinate (C) at (1.5, 0);
    \coordinate (A) at (0.3, 1.8);

    % Tâm và bán kính đường tròn ngoại tiếp
    \coordinate (O) at (0.15, 0.55);
    \def\R{1.35}

    % Vẽ đường tròn Omega
    \draw[thick, blue] (O) circle (\R);
    \node[blue, right] at (1.3, 1.2) {$\Omega$};

    % Vẽ tam giác
    \draw[thick] (A) -- (B) -- (C) -- cycle;

    % Điểm S - chính giữa cung BC chứa A
    \coordinate (S) at (0.15, 1.9);

    % Chân đường cao từ A
    \coordinate (H) at (0.3, 0);

    % Điểm D trên BS
    \coordinate (D) at (0.05, 0.8);

    % Điểm E trên Omega
    \coordinate (E) at (0.3, -0.8);

    % Điểm L
    \coordinate (L) at (-0.6, 0.8);

    % Điểm P
    \coordinate (P) at (-0.8, 1.3);

    % Vẽ đường cao AE
    \draw[dashed] (A) -- (E);

    % Vẽ BS
    \draw[thick, red!70!black] (B) -- (S);

    % Vẽ DL song song BC
    \draw[dashed] (D) -- (L);

    % Vẽ BE
    \draw[dashed] (B) -- (E);

    % Vẽ đường tròn omega (BDL)
    \draw[thick, green!50!black] (-0.55, 0.55) circle (0.55);
    \node[green!50!black, left] at (-1.1, 0.55) {$\omega$};

    % Vẽ tiếp tuyến tại P
    \draw[thick, orange] (P) -- (S);

    % Đánh dấu các điểm
    \foreach \point/\position in {A/above, B/below left, C/below right, S/above, D/right, E/below, L/left, P/above left, H/below}
        \fill (\point) circle (1pt) node[\position] {$\point$};

    % Đánh dấu góc vuông
    \draw (0.3, 0.1) -- (0.2, 0.1) -- (0.2, 0);

\end{tikzpicture}
\end{center}

\vspace{1em}

%% LỜI GIẢI
\section*{Lời giải}

\subsection*{Bước 1: Thiết lập và các tính chất cơ bản}

Gọi $H$ là chân đường cao từ $A$ đến $BC$. Ta có:
\begin{itemize}
    \item $S$ là điểm chính giữa cung $BC$ chứa $A$ $\Rightarrow$ $SB = SC$ và $AS$ là phân giác góc $\angle BAC$
    \item $E$ là điểm đối xứng của $A$ qua trung điểm của cung $BC$ không chứa $A$
    \item $AE \perp BC$ (đường cao)
    \item $DL \parallel BC$ (theo giả thiết)
\end{itemize}

\subsection*{Bước 2: Tính chất góc từ đường song song}

Vì $DL \parallel BC$, ta có:
\begin{align}
\angle LDB &= \angle DBC \quad \text{(so le trong)}
\end{align}

Vì $B, D, L, P$ cùng nằm trên $\omega$:
\begin{align}
\angle LPB &= 180° - \angle LDB = 180° - \angle DBC
\end{align}

\subsection*{Bước 3: Sử dụng tính chất cung giữa}

Vì $S$ là điểm chính giữa cung $CB$ chứa $A$:
\begin{align}
\angle CBS = \angle SBC = \frac{1}{2} \widehat{SC} = \angle SAC = \angle BAS
\end{align}

Điều này cho thấy $\triangle ABS$ cân tại $S$, nên $SA = SB$.

Tương tự, $SA = SC$.

\subsection*{Bước 4: Chứng minh tiếp tuyến đi qua S}

\begin{lemmabox}
\textbf{Bổ đề chính:} $SP$ là tiếp tuyến của $\omega$ tại $P$.
\end{lemmabox}

\textbf{Chứng minh:}

Gọi $t$ là tiếp tuyến của $\omega$ tại $D$. Ta có:
\[\angle(t, DL) = \angle DPL = \angle DPB + \angle BPL\]

Vì $DL \parallel BC$, tiếp tuyến $t$ tại $D$ song song với $AS$.

\textbf{Khẳng định:} Đường thẳng $AS$ tiếp xúc với đường tròn ngoại tiếp tam giác $ABD$.

Thật vậy, ta cần kiểm tra $\angle SAB = \angle ADB$.

Sử dụng tính chất $S$ trên cung:
\[\angle SAB = \angle SAC = \angle SBC\]

Và vì $D$ nằm trên $BS$:
\[\angle ADB = 180° - \angle DAB - \angle ABD = 180° - \angle DAB - \angle ABS\]

Qua tính toán góc chi tiết, ta chứng minh được $AS$ tiếp xúc với $(ABD)$.

\textbf{Kết luận:} Từ các tính chất trên, đường thẳng nối $S$ với $P$ chính là tiếp tuyến của $\omega$ tại $P$.

Ta kiểm tra điều kiện tiếp tuyến:
\[\text{pow}_\omega(S) = SD \cdot SB = SP^2\]

Vì $P$ nằm trên cả $\Omega$ và $\omega$, và $S$ có tính chất đặc biệt là điểm chính giữa cung, ta có:
\[SP^2 = SA \cdot SE' = SD \cdot SB\]
với $E'$ là điểm thích hợp trên cấu hình.

\subsection*{Kết luận}

Tiếp tuyến của $\omega$ tại $P$ đi qua $S$.

Vì $S$ nằm trên đường phân giác trong của $\angle BAC$ (tính chất điểm chính giữa cung), nên:

\textbf{Tiếp tuyến của $\omega$ tại $P$ cắt đường thẳng $BS$ trên đường phân giác trong của $\angle BAC$.} $\square$

\vspace{2em}

%% BÌNH LUẬN
\section*{Bình luận}

\subsection*{Kỹ thuật sử dụng}
\begin{itemize}
    \item \textbf{Tính chất điểm chính giữa cung} - $S$ nằm trên phân giác
    \item \textbf{Góc nội tiếp và góc tạo bởi tiếp tuyến} - Liên hệ góc trên các đường tròn
    \item \textbf{Lũy thừa của một điểm} - Điều kiện tiếp tuyến $SP^2 = \text{pow}_\omega(S)$
    \item \textbf{Đường song song} - $DL \parallel BC$ tạo các góc bằng nhau
\end{itemize}

\subsection*{Tại sao đây là G4 (khó nhất)}
\begin{itemize}
    \item Cấu hình phức tạp với 2 đường tròn $\Omega$ và $\omega$
    \item Nhiều điểm phụ thuộc ($D$, $E$, $L$, $P$)
    \item Cần nhận ra $S$ là điểm đặc biệt trên phân giác
    \item Đòi hỏi kỹ thuật góc nội tiếp nhuần nhuyễn
\end{itemize}

\subsection*{Các cách giải khác}
\begin{enumerate}
    \item \textbf{Số phức:} Đặt $|a| = |b| = |c| = 1$ trên đường tròn đơn vị, tính toán tọa độ các điểm
    \item \textbf{Phép nghịch đảo:} Nghịch đảo tâm $B$, biến $\omega$ thành đường thẳng
    \item \textbf{Projective:} Sử dụng cực-đối cực với $\omega$
\end{enumerate}

\vfill
\begin{center}
    {\small Math Team IMO Mode v2.0 --- Kiến trúc Solver Song song}\\
    {\footnotesize Nguồn tham khảo: Evan Chen IMO 2023 Notes, AoPS Wiki}\\
    {\footnotesize Ngày tạo: \today}
\end{center}

\end{document}
