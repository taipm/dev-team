% IMO 2023 Problem 1 - Math Team Solution
% Generated by Math Team IMO Mode v2.0

\documentclass[11pt,a4paper]{article}

% XeLaTeX Vietnamese support
\usepackage{fontspec}
\usepackage[vietnamese]{babel}
\setmainfont{Times New Roman}

% Math packages
\usepackage{amsmath,amssymb,amsthm}
\usepackage{mathtools}

% Layout
\usepackage{geometry}
\geometry{margin=2.5cm}
\usepackage{xcolor}
\usepackage{tcolorbox}
\usepackage{array}
\usepackage{booktabs}

% Colors
\definecolor{trickcolor}{RGB}{0, 100, 180}
\definecolor{notecolor}{RGB}{180, 100, 0}
\definecolor{proofcolor}{RGB}{0, 120, 60}

% Custom boxes
\newtcolorbox{problembox}{
    colback=gray!10,
    colframe=black,
    title={\textbf{Problem 1}},
    fonttitle=\bfseries\large
}

\newtcolorbox{trickbox}{
    colback=trickcolor!10,
    colframe=trickcolor,
    title={\textbf{Key Insight}},
    fonttitle=\bfseries
}

\newtcolorbox{answerbox}{
    colback=proofcolor!10,
    colframe=proofcolor,
    fonttitle=\bfseries
}

% Header
\usepackage{fancyhdr}
\pagestyle{fancy}
\fancyhf{}
\rhead{{\small Difficulty: IMO P1}}
\lhead{{\small Number Theory}}
\cfoot{\thepage}

% Theorems
\newtheorem{theorem}{Theorem}
\newtheorem{lemma}[theorem]{Lemma}
\newtheorem{claim}[theorem]{Claim}

\begin{document}

\begin{center}
    {\LARGE\bfseries IMO 2023 Problem 1}\\[0.3em]
    {\large 64th International Mathematical Olympiad}\\[0.2em]
    {\small Chiba, Japan | July 2023}
\end{center}

\vspace{0.5em}
\hrule
\vspace{1em}

%% PROBLEM
\begin{problembox}
Determine all composite integers $n > 1$ that satisfy the following property: if $d_1, d_2, \ldots, d_k$ are all the positive divisors of $n$ with
\[1 = d_1 < d_2 < \cdots < d_k = n,\]
then $d_i$ divides $d_{i+1} + d_{i+2}$ for every $1 \leq i \leq k-2$.
\end{problembox}

\vspace{1em}

%% KEY INSIGHT
\begin{trickbox}
\textbf{Main Observation:} The divisor structure of prime powers $p^a$ forms a geometric progression $\{1, p, p^2, \ldots, p^a\}$, making the divisibility condition automatic:
\[d_i = p^{i-1} \mid p^i + p^{i+1} = p^i(1+p) = d_{i+1}(1+p)\]

For numbers with $\geq 2$ distinct primes, there's always a ``transition point'' where the condition fails.
\end{trickbox}

\vspace{1em}

%% ANSWER
\begin{answerbox}
\textbf{Answer:} $\boxed{n = p^a \text{ where } p \text{ is prime and } a \geq 2}$

All composite integers satisfying the condition are exactly the \textbf{prime powers} with exponent at least 2.
\end{answerbox}

\vspace{1.5em}

%% SOLUTION
\section*{Solution}

\subsection*{Step 1: Exploration with Small Cases}

\begin{center}
\begin{tabular}{c|c|l|c}
\toprule
$n$ & Prime Power? & Divisors & Works? \\
\midrule
$4 = 2^2$ & Yes & $1, 2, 4$ & $\checkmark$ \\
$6 = 2 \cdot 3$ & No & $1, 2, 3, 6$ & $\times$ $(2 \nmid 9)$ \\
$8 = 2^3$ & Yes & $1, 2, 4, 8$ & $\checkmark$ \\
$9 = 3^2$ & Yes & $1, 3, 9$ & $\checkmark$ \\
$10 = 2 \cdot 5$ & No & $1, 2, 5, 10$ & $\times$ $(2 \nmid 15)$ \\
$12 = 2^2 \cdot 3$ & No & $1, 2, 3, 4, 6, 12$ & $\times$ $(2 \nmid 7)$ \\
$16 = 2^4$ & Yes & $1, 2, 4, 8, 16$ & $\checkmark$ \\
$25 = 5^2$ & Yes & $1, 5, 25$ & $\checkmark$ \\
$27 = 3^3$ & Yes & $1, 3, 9, 27$ & $\checkmark$ \\
\bottomrule
\end{tabular}
\end{center}

\textbf{Conjecture:} The answer is exactly the prime powers $n = p^a$ with $a \geq 2$.

\subsection*{Step 2: Prime Powers Satisfy the Condition}

\begin{theorem}
For any prime $p$ and integer $a \geq 2$, the number $n = p^a$ satisfies the given property.
\end{theorem}

\begin{proof}
The divisors of $n = p^a$ are exactly $\{1, p, p^2, \ldots, p^a\}$, so $d_j = p^{j-1}$ for $j = 1, 2, \ldots, a+1$.

For any $1 \leq i \leq a-1$, we verify $d_i \mid d_{i+1} + d_{i+2}$:
\begin{align*}
d_{i+1} + d_{i+2} &= p^i + p^{i+1} = p^i(1 + p)
\end{align*}

Since $d_i = p^{i-1}$ divides $p^i = p \cdot p^{i-1}$, we have:
\[d_i = p^{i-1} \mid p^i(1+p) = d_{i+1} + d_{i+2}\]

Therefore the condition holds for all valid $i$.
\end{proof}

\subsection*{Step 3: Non-Prime-Powers Fail the Condition}

\begin{theorem}
If $n$ has at least two distinct prime divisors, then $n$ fails the condition.
\end{theorem}

\begin{proof}
Let $p < q$ be the two smallest distinct prime divisors of $n$. The divisor sequence begins:
\[d_1 = 1, \quad d_2 = p\]

The key is analyzing what happens at $d_3$ and $d_4$.

\textbf{Case 1:} $q < p^2$ (so $d_3 = q$)

Then $d_4 \geq \min(p^2, pq)$.

If $d_4 = p^2$: We need $p \mid d_3 + d_4 = q + p^2$, which requires $p \mid q$. But $q$ is prime and $q > p$, so $p \nmid q$. \textbf{Contradiction.}

If $d_4 = pq$: We need $p \mid q + pq = q(1+p)$. Since $\gcd(p,q) = 1$ and $p \nmid (1+p)$, this fails. \textbf{Contradiction.}

\textbf{Case 2:} $p^2 \leq q$ (so $d_3 = p^2$ if $p^2 \mid n$)

Eventually, $q$ must appear in the divisor list. Let $d_j = p^m$ and $d_{j+1} = q$ for some $m$.

At $i = j-1$: $d_i = p^{m-1}$, and we need $p^{m-1} \mid p^m + q$.

Since $p^{m-1} \mid p^m$, we need $p^{m-1} \mid q$. But $q$ is prime and $q \neq p$. \textbf{Contradiction.}

In all cases, non-prime-powers fail the condition.
\end{proof}

\subsection*{Conclusion}

The complete set of composite integers satisfying the property is:
\[\boxed{n = p^a \text{ where } p \text{ is a prime and } a \geq 2}\]

Examples: $4, 8, 9, 16, 25, 27, 32, 49, 64, 81, 121, 125, \ldots$

\vspace{2em}

%% COMMENTARY
\section*{Commentary}

\subsection*{Techniques Used}
\begin{itemize}
    \item \textbf{Small cases exploration} - Testing $n = 4, 6, 8, \ldots$ to identify pattern
    \item \textbf{Divisibility chain analysis} - Studying the structure $d_i \mid d_{i+1} + d_{i+2}$
    \item \textbf{Proof by contradiction} - Showing non-prime-powers fail
    \item \textbf{Prime factorization structure} - Key insight about geometric vs non-geometric divisor sequences
\end{itemize}

\subsection*{Why This is an IMO P1}
\begin{itemize}
    \item Accessible entry point (small cases)
    \item Clean answer (prime powers)
    \item Requires careful case analysis but not deep theory
    \item Tests mathematical maturity more than specialized knowledge
\end{itemize}

\subsection*{Generalization}

\textbf{Question:} For which $n$ does $d_i \mid d_{i+1} + d_{i+2} + \cdots + d_{i+m}$ hold for all valid $i$?

\textbf{Conjecture:} The answer is still prime powers, since the geometric structure of $\{1, p, p^2, \ldots\}$ makes any sum divisible by earlier terms.

\vfill
\begin{center}
    {\small Math Team IMO Mode v2.0 --- Parallel Solver Architecture}\\
    {\footnotesize Generated: \today}
\end{center}

\end{document}
