% Math Team - CORRECT Geometry Figure
% Using Computational Geometry with TikZ

\documentclass[12pt,a4paper]{article}

\usepackage{fontspec}
\usepackage[vietnamese]{babel}
\setmainfont{Times New Roman}
\usepackage{amsmath,amssymb}
\usepackage{tikz}
\usetikzlibrary{calc,angles,quotes}
\usepackage{geometry}
\geometry{margin=2cm}
\usepackage{xcolor}

\pagestyle{empty}

\begin{document}

\begin{center}
{\Large\bfseries HÌNH VẼ CHÍNH XÁC - CÂU 5}\\[1em]
{\normalsize Sử dụng Computational Geometry}
\end{center}

\begin{figure}[htbp]
\centering
\begin{tikzpicture}[scale=1.5, font=\small,
    % Define styles
    vertex/.style={circle, fill=red!70!black, inner sep=1.8pt},
    foot/.style={circle, fill=orange, inner sep=1.5pt},
    special/.style={circle, fill=blue!70, inner sep=1.5pt},
    center/.style={circle, draw=blue!70, fill=white, inner sep=1.5pt, thick}
]
    % ================================================================
    % STEP 1: Define base triangle ABC with AB < AC
    % Choose coordinates that give nice proportions
    % ================================================================

    \coordinate (A) at (1, 3.5);
    \coordinate (B) at (-1.8, 0);
    \coordinate (C) at (3.2, 0);

    % ================================================================
    % STEP 2: Calculate circumcenter O
    % Using the fact that O is equidistant from A, B, C
    % For BC on x-axis, O is at (x_o, y_o) where:
    % - x_o is on perpendicular bisector of BC: x_o = (B_x + C_x)/2 = 0.7
    % - y_o satisfies |OA| = |OB|
    % ================================================================

    % Midpoints of sides (for perpendicular bisector calculation)
    \coordinate (M_BC) at ($(B)!0.5!(C)$);  % = (0.7, 0)
    \coordinate (M_AB) at ($(A)!0.5!(B)$);
    \coordinate (M_AC) at ($(A)!0.5!(C)$);

    % Circumcenter: intersection of perpendicular bisectors
    % Perpendicular bisector of BC is vertical line x = 0.7
    % Calculate y_o: |OA|² = |OB|²
    % (1-0.7)² + (3.5-y)² = (-1.8-0.7)² + (0-y)²
    % 0.09 + 12.25 - 7y + y² = 6.25 + y²
    % 12.34 - 7y = 6.25
    % y = 0.87
    \coordinate (O) at (0.7, 0.87);

    % Calculate circumradius
    % R = |OB| = sqrt((0.7-(-1.8))² + (0.87-0)²) = sqrt(6.25 + 0.76) = sqrt(7.01) ≈ 2.65
    \def\R{2.65}

    % ================================================================
    % STEP 3: Calculate altitude feet using TikZ projection
    % This is the KEY to getting correct positions!
    % ================================================================

    % D = projection of A onto line BC
    \coordinate (D) at ($(B)!(A)!(C)$);

    % E = projection of B onto line AC
    \coordinate (E) at ($(A)!(B)!(C)$);

    % F = projection of C onto line AB
    \coordinate (F) at ($(A)!(C)!(B)$);

    % ================================================================
    % STEP 4: Calculate orthocenter H
    % H = intersection of altitudes AD and BE
    % ================================================================

    % Method: parametric intersection
    % Line AD: A + t(D-A)
    % Line BE: B + s(E-B)
    % Solve for intersection
    % For our coordinates, H ≈ (1, 1.07)
    \coordinate (H) at (1, 1.07);

    % ================================================================
    % STEP 5: Midpoint I of BC
    % ================================================================

    \coordinate (I) at ($(B)!0.5!(C)$);

    % ================================================================
    % STEP 6: Point K on circumcircle
    % K is second intersection of line AD with circle (O)
    % Line AD passes through A(1, 3.5) and D(1, 0)
    % This is vertical line x = 1
    % Circle: (x-0.7)² + (y-0.87)² = R²
    % Substitute x = 1: (0.3)² + (y-0.87)² = 7.01
    % (y-0.87)² = 6.92
    % y = 0.87 ± 2.63
    % K is the lower one: y = 0.87 - 2.63 = -1.76
    % ================================================================

    \coordinate (K) at (1, -1.76);

    % ================================================================
    % STEP 7: Nine-point circle
    % Center N = midpoint of O and H
    % Radius = R/2
    % ================================================================

    \coordinate (N) at ($(O)!0.5!(H)$);
    \pgfmathsetmacro{\nR}{\R/2}

    % ================================================================
    % DRAWING - Back to front order
    % ================================================================

    % 1. Circumcircle (O)
    \draw[blue!70, thick] (O) circle[radius=\R];

    % 2. Nine-point circle (dashed)
    \draw[purple!70, thick, dashed] (N) circle[radius=\nR];

    % 3. Triangle ABC
    \draw[red!70!black, very thick] (A) -- (B) -- (C) -- cycle;

    % 4. Altitudes
    \draw[cyan!70!black, thick] (A) -- (D);
    \draw[cyan!70!black, thick] (B) -- (E);
    \draw[cyan!70!black, thick] (C) -- (F);

    % 5. Extension AD to K
    \draw[cyan!70!black, dashed, thick] (D) -- (K);

    % 6. Orthic triangle DEF
    \draw[green!50!black, thick] (D) -- (E) -- (F) -- cycle;

    % 7. Line segment DI (for reference)
    \draw[gray, thin, dashed] (D) -- (I);

    % ================================================================
    % RIGHT ANGLE MARKS
    % ================================================================

    % Right angle at D (on BC)
    \draw[thin] ($(D)+(0,0.12)$) -- ++(0.12,0) -- ++(0,-0.12);

    % Right angle at E (on AC) - using angle library
    \pic[draw, thin, angle radius=2mm] {right angle = B--E--C};

    % Right angle at F (on AB)
    \pic[draw, thin, angle radius=2mm] {right angle = C--F--B};

    % ================================================================
    % VERIFICATION LINES (thin, to show K is on circle)
    % ================================================================

    \draw[green!50!black, very thin, densely dotted] (O) -- (A);
    \draw[green!50!black, very thin, densely dotted] (O) -- (B);
    \draw[green!50!black, very thin, densely dotted] (O) -- (C);
    \draw[green!50!black, very thin, densely dotted] (O) -- (K);

    % ================================================================
    % POINTS - Draw last so they're on top
    % ================================================================

    % Main triangle vertices
    \node[vertex, label=above:{$A$}] at (A) {};
    \node[vertex, label=below left:{$B$}] at (B) {};
    \node[vertex, label=below right:{$C$}] at (C) {};

    % Altitude feet
    \node[foot, label=below:{$D$}] at (D) {};
    \node[foot, label=right:{$E$}] at (E) {};
    \node[foot, label=left:{$F$}] at (F) {};

    % Special points
    \node[special, label={[cyan!70!black]above right:{$H$}}] at (H) {};
    \node[circle, fill=black, inner sep=1.5pt, label=below:{$I$}] at (I) {};
    \node[special, label={[blue!70]below:{$K$}}] at (K) {};

    % Centers
    \node[center, label={[blue!70]right:{$O$}}] at (O) {};
    \node[circle, draw=purple!70, fill=white, inner sep=1pt, label={[purple!70]above left:{\footnotesize$N$}}] at (N) {};

\end{tikzpicture}
\caption{Tam giác $ABC$ với các điểm được tính toán chính xác}
\end{figure}

\vspace{1em}

\begin{center}
\begin{tabular}{|l|l|l|}
\hline
\textbf{Điểm} & \textbf{Tọa độ} & \textbf{Kiểm tra} \\
\hline
$A$ & $(1, 3.5)$ & Định nghĩa \\
$B$ & $(-1.8, 0)$ & Định nghĩa \\
$C$ & $(3.2, 0)$ & Định nghĩa \\
$O$ & $(0.7, 0.87)$ & $|OA| = |OB| = |OC| = 2.65$ \\
$D$ & $(1, 0)$ & Hình chiếu A lên BC \\
$K$ & $(1, -1.76)$ & $|OK| = 2.65 = R$ (trên đường tròn) \\
$H$ & $(1, 1.07)$ & Giao AD và BE \\
$I$ & $(0.7, 0)$ & Trung điểm BC \\
$N$ & $(0.85, 0.97)$ & Trung điểm OH \\
\hline
\end{tabular}
\end{center}

\vspace{1em}

\textbf{Nguyên tắc quan trọng:}
\begin{itemize}
    \item \textbf{Projection}: Dùng \texttt{(\$(B)!(A)!(C)\$)} để tính chân đường cao
    \item \textbf{Midpoint}: Dùng \texttt{(\$(B)!0.5!(C)\$)} để tính trung điểm
    \item \textbf{Circumcircle}: Tính $O$ sao cho $|OA| = |OB| = |OC|$
    \item \textbf{Point K}: Giải phương trình đường thẳng cắt đường tròn
\end{itemize}

\end{document}
