% Math Team - Solution Document
% Generated by LaTeX Editor Agent
% Style: detailed (textbook)

\documentclass[12pt,a4paper]{article}

% XeLaTeX Vietnamese support
\usepackage{fontspec}
\usepackage[vietnamese]{babel}
\setmainfont{Times New Roman}
\setsansfont{Arial}
\setmonofont{Courier New}

% Math packages
\usepackage{amsmath,amssymb,amsthm}
\usepackage{mathtools}

% Layout
\usepackage{geometry}
\geometry{margin=2.5cm}
\usepackage{enumitem}
\usepackage{xcolor}
\usepackage{tcolorbox}

% Theorem environments (Vietnamese)
\theoremstyle{definition}
\newtheorem{theorem}{Định lý}[section]
\theoremstyle{remark}
\newtheorem{remark}{Nhận xét}

% Custom box
\newtcolorbox{answerbox}{
    colback=blue!5,
    colframe=blue!50!black,
    title={\textbf{Đáp số}},
    fonttitle=\bfseries
}

% Header
\usepackage{fancyhdr}
\pagestyle{fancy}
\fancyhf{}
\rhead{Math Team}
\lhead{Lời giải chi tiết}
\cfoot{\thepage}

\begin{document}

\begin{center}
    {\LARGE\bfseries LỜI GIẢI CHI TIẾT}\\[0.5em]
    {\large Math Team - Signal-Based Problem Solving}\\[0.5em]
    {\small Câu 3a - Đề thi THPT (1.5 điểm)}\\[1em]
    {\small Ngày tạo: \today}
\end{center}

\hrule
\vspace{1em}

\section{Đề bài}

Giải phương trình:
\[
2(x+1) + \frac{1}{x-1} + \frac{2}{x-2} = 0
\]

\subsection*{Phân tích đề bài}
\begin{itemize}
    \item \textbf{Cho:} Phương trình chứa phân thức với các mẫu $(x-1)$ và $(x-2)$
    \item \textbf{Tìm:} Giá trị $x$ thỏa mãn phương trình
    \item \textbf{Loại:} Đại số - Phương trình phân thức
    \item \textbf{Độ khó:} THPT
\end{itemize}

\section{Lời giải}

\subsection{Bước 1: Xác định điều kiện xác định}

Phương trình chứa các phân thức với mẫu số $(x-1)$ và $(x-2)$.

Điều kiện xác định:
\begin{align*}
x - 1 &\neq 0 \Rightarrow x \neq 1 \\
x - 2 &\neq 0 \Rightarrow x \neq 2
\end{align*}

\textbf{Điều kiện:} $x \in \mathbb{R} \setminus \{1, 2\}$

\subsection{Bước 2: Đặt ẩn phụ}

Đặt $u = x - 1$, khi đó:
\begin{itemize}
    \item $x = u + 1$
    \item $x + 1 = u + 2$
    \item $x - 2 = u - 1$
\end{itemize}

Phương trình trở thành:
\[
2(u + 2) + \frac{1}{u} + \frac{2}{u-1} = 0
\]

Hay:
\[
2u + 4 + \frac{1}{u} + \frac{2}{u-1} = 0
\]

\subsection{Bước 3: Quy đồng mẫu số}

Nhân cả hai vế với $u(u-1)$ (với $u \neq 0$ và $u \neq 1$):

\[
(2u + 4) \cdot u(u-1) + (u-1) + 2u = 0
\]

Khai triển $(2u + 4) \cdot u(u-1)$:
\begin{align*}
(2u + 4)(u^2 - u) &= 2u^3 - 2u^2 + 4u^2 - 4u \\
&= 2u^3 + 2u^2 - 4u
\end{align*}

Phương trình trở thành:
\[
2u^3 + 2u^2 - 4u + u - 1 + 2u = 0
\]

\subsection{Bước 4: Rút gọn}

Thu gọn các hạng tử đồng dạng:
\[
2u^3 + 2u^2 + (-4u + u + 2u) - 1 = 0
\]
\[
2u^3 + 2u^2 - u - 1 = 0
\]

\subsection{Bước 5: Phân tích thành nhân tử}

Nhóm các số hạng:
\begin{align*}
2u^3 + 2u^2 - u - 1 &= 2u^2(u + 1) - (u + 1) \\
&= (u + 1)(2u^2 - 1)
\end{align*}

Phương trình trở thành:
\[
(u + 1)(2u^2 - 1) = 0
\]

\begin{remark}
Đây là điểm hay của cách đặt ẩn phụ - việc phân tích nhân tử trở nên đơn giản hơn so với cách quy đồng trực tiếp.
\end{remark}

\subsection{Bước 6: Giải phương trình}

\textbf{Trường hợp 1:} $u + 1 = 0$
\[
u = -1 \Rightarrow x = u + 1 = 0
\]

\textbf{Trường hợp 2:} $2u^2 - 1 = 0$
\[
u^2 = \frac{1}{2} \Rightarrow u = \pm\frac{1}{\sqrt{2}} = \pm\frac{\sqrt{2}}{2}
\]

Suy ra:
\begin{align*}
u = \frac{\sqrt{2}}{2} &\Rightarrow x = 1 + \frac{\sqrt{2}}{2} = \frac{2 + \sqrt{2}}{2} \\
u = -\frac{\sqrt{2}}{2} &\Rightarrow x = 1 - \frac{\sqrt{2}}{2} = \frac{2 - \sqrt{2}}{2}
\end{align*}

\subsection{Bước 7: Kiểm tra điều kiện xác định}

Các nghiệm tìm được:
\begin{itemize}
    \item $x_1 = 0$: Thỏa mãn ($0 \neq 1$ và $0 \neq 2$) \checkmark
    \item $x_2 = \frac{2 + \sqrt{2}}{2} \approx 1.707$: Thỏa mãn \checkmark
    \item $x_3 = \frac{2 - \sqrt{2}}{2} \approx 0.293$: Thỏa mãn \checkmark
\end{itemize}

\subsection{Bước 8: Kiểm tra nghiệm}

\textbf{Kiểm tra $x = 0$:}
\[
2(0+1) + \frac{1}{0-1} + \frac{2}{0-2} = 2 - 1 - 1 = 0 \quad \checkmark
\]

\section{Kết luận}

\begin{answerbox}
Phương trình có 3 nghiệm:
\[
\boxed{x \in \left\{ 0, \; \frac{2 - \sqrt{2}}{2}, \; \frac{2 + \sqrt{2}}{2} \right\}}
\]

Hay viết dạng khác:
\[
\boxed{x = 0 \quad \text{hoặc} \quad x = 1 \pm \frac{\sqrt{2}}{2}}
\]
\end{answerbox}

\section*{Ghi nhớ}

\begin{itemize}
    \item Khi gặp phương trình phân thức, luôn xác định ĐKXĐ trước
    \item Đặt ẩn phụ $u = x - a$ với $a$ phù hợp có thể giúp đơn giản hóa việc phân tích nhân tử
    \item Phương trình bậc 3 dạng $au^3 + bu^2 + cu + d = 0$ thường phân tích được bằng cách nhóm
\end{itemize}

\vfill
\hrule
\begin{center}
    {\small Được tạo bởi Math Team - Signal-Based Problem Solving}\\
    {\small Session: session-001}
\end{center}

\end{document}
