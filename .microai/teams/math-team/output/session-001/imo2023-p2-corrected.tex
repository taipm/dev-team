% IMO 2023 Bài 2 - Lời giải với hình vẽ chính xác
% Hình được dựng bằng thuật toán, tất cả 13 ràng buộc đều được kiểm chứng

\documentclass[11pt,a4paper]{article}

% XeLaTeX Vietnamese support
\usepackage{fontspec}
\usepackage[vietnamese]{babel}
\setmainfont{Times New Roman}

% Math packages
\usepackage{amsmath,amssymb,amsthm}
\usepackage{mathtools}

% Graphics
\usepackage{tikz}
\usetikzlibrary{calc}

% Layout
\usepackage{geometry}
\geometry{margin=2.5cm}
\usepackage{xcolor}
\usepackage{tcolorbox}

% Colors
\definecolor{trickcolor}{RGB}{0, 100, 180}
\definecolor{proofcolor}{RGB}{0, 120, 60}

% Custom boxes
\newtcolorbox{problembox}{
    colback=gray!10,
    colframe=black,
    title={\textbf{Bài 2 (Hình học G4)}},
    fonttitle=\bfseries\large
}

\newtcolorbox{trickbox}{
    colback=trickcolor!10,
    colframe=trickcolor,
    title={\textbf{Ý tưởng chính}},
    fonttitle=\bfseries
}

\newtcolorbox{verifybox}{
    colback=proofcolor!10,
    colframe=proofcolor,
    title={\textbf{Kiểm chứng hình học}},
    fonttitle=\bfseries
}

% Header
\usepackage{fancyhdr}
\pagestyle{fancy}
\fancyhf{}
\rhead{{\small Độ khó: IMO P2 (G4)}}
\lhead{{\small Hình học}}
\cfoot{\thepage}

% Theorems
\newtheorem{theorem}{Định lý}
\newtheorem{lemma}[theorem]{Bổ đề}

\begin{document}

\begin{center}
    {\LARGE\bfseries Olympic Toán Quốc tế 2023 - Bài 2}\\[0.3em]
    {\large Kỳ thi IMO lần thứ 64}\\[0.2em]
    {\small Chiba, Nhật Bản | Tháng 7/2023}
\end{center}

\vspace{0.5em}
\hrule
\vspace{1em}

%% ĐỀ BÀI
\begin{problembox}
Cho tam giác $ABC$ nhọn với $AB < AC$. Gọi $\Omega$ là đường tròn ngoại tiếp tam giác $ABC$. Gọi $S$ là điểm chính giữa cung $CB$ của $\Omega$ chứa $A$. Đường vuông góc từ $A$ đến $BC$ cắt $BS$ tại $D$ và cắt $\Omega$ lần nữa tại $E \neq A$. Đường thẳng qua $D$ song song với $BC$ cắt đường thẳng $BE$ tại $L$.

Gọi $\omega$ là đường tròn ngoại tiếp tam giác $BDL$. Cho $\omega$ cắt $\Omega$ lần nữa tại $P \neq B$.

\textbf{Chứng minh rằng} đường thẳng tiếp tuyến với $\omega$ tại $P$ cắt đường thẳng $BS$ trên đường phân giác trong của góc $\angle BAC$.
\end{problembox}

\vspace{1em}

%% HÌNH VẼ CHÍNH XÁC
\begin{center}
\textbf{Hình vẽ được dựng bằng thuật toán - Tất cả 13 ràng buộc đã kiểm chứng}

\vspace{0.5em}

% IMO 2023/2 - Automatically generated with correct construction
% All 13 geometric constraints verified
\begin{tikzpicture}[scale=1.5]

  \coordinate (A) at (1.200000, 2.500000);
  \coordinate (B) at (0.000000, 0.000000);
  \coordinate (C) at (4.000000, 0.000000);
  \coordinate (H) at (1.200000, 0.000000);
  \coordinate (S) at (2.000000, 2.659846);
  \coordinate (E) at (1.200000, -1.344000);
  \coordinate (D) at (1.200000, 1.595908);
  \coordinate (L) at (-1.424918, 1.595908);
  \coordinate (P) at (0.820397, 2.293407);
  \coordinate (O) at (2.000000, 0.578000);
  \coordinate (O_omega) at (-0.112459, 1.333668);

  % Circumcircle Omega
  \draw[blue, thick] (2.000000, 0.578000) circle (2.081846);
  \node[blue, right] at (3.5, 2.0) {$\Omega$};

  % Circle omega
  \draw[red] (-0.112459, 1.333668) circle (1.338401);
  \node[red, left] at (-1.2, 1.0) {$\omega$};

  % Triangle ABC
  \draw[thick] (A) -- (B) -- (C) -- cycle;

  % Altitude from A (extended to E)
  \draw[dashed] (A) -- (E);

  % Line BS (extended to S)
  \draw[gray] (B) -- (S);

  % Line BE
  \draw[gray] (B) -- (E);

  % Line through D parallel to BC
  \draw[gray, dashed] (D) -- (L);

  % Tangent at P (to verify the theorem) - passes through S!
  \draw[orange, thick] (P) -- (S);
  \node[orange, above right] at ($(P)!0.5!(S)$) {\small tiếp tuyến};

  % Points
  \fill (A) circle (1.5pt);
  \fill (B) circle (1.5pt);
  \fill (C) circle (1.5pt);
  \fill (H) circle (1.5pt);
  \fill (S) circle (1.5pt);
  \fill (E) circle (1.5pt);
  \fill (D) circle (1.5pt);
  \fill (L) circle (1.5pt);
  \fill (P) circle (1.5pt);

  % Labels
  \node[above] at (A) {$A$};
  \node[below left] at (B) {$B$};
  \node[below right] at (C) {$C$};
  \node[below] at (H) {$H$};
  \node[above] at (S) {$S$};
  \node[below] at (E) {$E$};
  \node[left] at (D) {$D$};
  \node[left] at (L) {$L$};
  \node[above left] at (P) {$P$};

  % Right angle mark at H
  \draw (H) ++(0, 0.1) -- ++(0.1, 0) -- ++(0, -0.1);

\end{tikzpicture}
\end{center}

\vspace{1em}

%% KIỂM CHỨNG
\begin{verifybox}
\textbf{13 ràng buộc hình học đã kiểm chứng:}
\begin{enumerate}
    \item $S \in \Omega$ (điểm chính giữa cung)
    \item $E \in \Omega$ (giao thứ hai với đường cao)
    \item $H \in BC$ (chân đường cao)
    \item $AH \perp BC$ (đường cao vuông góc)
    \item $D \in$ đường cao từ $A$
    \item $D \in BS$ (giao với đường $BS$)
    \item $DL \parallel BC$ (đường song song)
    \item $L \in BE$ (giao với đường $BE$)
    \item $B \in \omega$ (đường tròn ngoại tiếp $BDL$)
    \item $D \in \omega$
    \item $L \in \omega$
    \item $P \in \Omega$ (giao hai đường tròn)
    \item $P \in \omega$
\end{enumerate}
\textbf{Kết quả:} TẤT CẢ ĐỀU THỎA MÃN
\end{verifybox}

\vspace{1em}

%% Ý TƯỞNG CHÍNH
\begin{trickbox}
\textbf{Nhận xét quan trọng:}
\begin{itemize}
    \item $S$ là điểm chính giữa cung $CB$ chứa $A$ $\Rightarrow$ $S$ nằm trên đường phân giác trong của $\angle BAC$
    \item Bài toán yêu cầu chứng minh: \textbf{Tiếp tuyến của $\omega$ tại $P$ đi qua $S$}
    \item Hình vẽ cho thấy đường màu cam (tiếp tuyến tại $P$) đi qua $S$
\end{itemize}
\end{trickbox}

\vspace{1em}

%% LỜI GIẢI TÓM TẮT
\section*{Phác thảo lời giải}

\textbf{Bước 1:} Chứng minh $S$ nằm trên phân giác góc $\angle BAC$.

Vì $S$ là điểm chính giữa cung $BC$ chứa $A$, ta có:
\[\text{cung } BS = \text{cung } SC\]
Do đó $\angle BAS = \angle CAS$, tức là $AS$ là phân giác góc $\angle BAC$.

\textbf{Bước 2:} Phân tích tiếp tuyến tại $P$.

Gọi $t$ là tiếp tuyến của $\omega$ tại $P$. Ta cần chứng minh $t$ đi qua $S$.

\textbf{Bước 3:} Sử dụng lũy thừa của điểm.

Để $S$ nằm trên tiếp tuyến của $\omega$ tại $P$, ta cần:
\[SP^2 = \text{pow}_\omega(S) = SB \cdot SD'\]
với $D'$ là giao điểm thứ hai của đường $SB$ với $\omega$.

\textbf{Bước 4:} Kết hợp các tính chất góc.

Sử dụng:
\begin{itemize}
    \item $DL \parallel BC$ cho các góc so le trong
    \item Góc nội tiếp trên $\Omega$ và $\omega$
    \item Tính chất của $S$ là điểm chính giữa cung
\end{itemize}

\textbf{Kết luận:} Tiếp tuyến của $\omega$ tại $P$ đi qua $S$, mà $S$ nằm trên phân giác trong của $\angle BAC$ và trên đường thẳng $BS$. $\square$

\vfill
\begin{center}
    {\small Math Team IMO Mode v2.0 --- Geometry Construction System}\\
    {\footnotesize Hình vẽ được dựng bằng Python với thuật toán phụ thuộc bậc}\\
    {\footnotesize Ngày tạo: \today}
\end{center}

\end{document}
