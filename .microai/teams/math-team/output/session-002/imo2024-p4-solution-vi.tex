% IMO 2024 Bài 4 - Lời giải với hình vẽ chính xác
% Hình được dựng bằng Geometry System v2.0, tất cả ràng buộc đều được kiểm chứng

\documentclass[11pt,a4paper]{article}

% XeLaTeX Vietnamese support
\usepackage{fontspec}
\usepackage[vietnamese]{babel}
\setmainfont{Times New Roman}

% Math packages
\usepackage{amsmath,amssymb,amsthm}
\usepackage{mathtools}

% Graphics
\usepackage{tikz}
\usetikzlibrary{calc}

% Layout
\usepackage{geometry}
\geometry{margin=2.5cm}
\usepackage{xcolor}
\usepackage{tcolorbox}

% Colors
\definecolor{trickcolor}{RGB}{0, 100, 180}
\definecolor{proofcolor}{RGB}{0, 120, 60}

% Custom boxes
\newtcolorbox{problembox}{
    colback=gray!10,
    colframe=black,
    title={\textbf{Bài 4 (Hình học)}},
    fonttitle=\bfseries\large
}

\newtcolorbox{trickbox}{
    colback=trickcolor!10,
    colframe=trickcolor,
    title={\textbf{Ý tưởng chính}},
    fonttitle=\bfseries
}

\newtcolorbox{verifybox}{
    colback=proofcolor!10,
    colframe=proofcolor,
    title={\textbf{Kiểm chứng hình học}},
    fonttitle=\bfseries
}

% Header
\usepackage{fancyhdr}
\pagestyle{fancy}
\fancyhf{}
\rhead{{\small Độ khó: IMO P4}}
\lhead{{\small Hình học}}
\cfoot{\thepage}

% Theorems
\newtheorem{theorem}{Định lý}
\newtheorem{lemma}[theorem]{Bổ đề}
\newtheorem{claim}[theorem]{Khẳng định}

\begin{document}

\begin{center}
    {\LARGE\bfseries Olympic Toán Quốc tế 2024 - Bài 4}\\[0.3em]
    {\large Kỳ thi IMO lần thứ 65}\\[0.2em]
    {\small Bath, Vương quốc Anh | Tháng 7/2024}
\end{center}

\vspace{0.5em}
\hrule
\vspace{1em}

%% ĐỀ BÀI
\begin{problembox}
Cho tam giác $ABC$ với $AB < AC < BC$. Gọi $I$ và $\omega$ lần lượt là tâm nội tiếp và đường tròn nội tiếp của tam giác $ABC$. Gọi $X$ là điểm trên đường thẳng $BC$ khác $C$ sao cho đường thẳng qua $X$ song song với $AC$ tiếp xúc với $\omega$. Tương tự, gọi $Y$ là điểm trên đường thẳng $BC$ khác $B$ sao cho đường thẳng qua $Y$ song song với $AB$ tiếp xúc với $\omega$.

Gọi $AI$ cắt đường tròn ngoại tiếp tam giác $ABC$ lần nữa tại $P \neq A$. Gọi $K$ và $L$ lần lượt là trung điểm của $AC$ và $AB$.

\textbf{Chứng minh rằng} $\angle KIL + \angle YPX = 180°$.
\end{problembox}

\vspace{1em}

%% HÌNH VẼ CHÍNH XÁC
\begin{center}
\textbf{Hình vẽ được dựng bằng Geometry System v2.0 - Tất cả 10 ràng buộc đã kiểm chứng}

\vspace{0.5em}

% IMO 2024/4 - Automatically generated with Geometry System v2.0
% All geometric constraints verified
\begin{tikzpicture}[scale=1.5]

  \coordinate (A) at (1.800000, 2.400000);
  \coordinate (B) at (0.000000, 0.000000);
  \coordinate (C) at (5.000000, 0.000000);
  \coordinate (I) at (2.000000, 1.000000);
  \coordinate (O) at (2.500000, 0.000000);
  \coordinate (K) at (3.400000, 1.200000);
  \coordinate (L) at (0.900000, 1.200000);
  \coordinate (P) at (2.500000, -2.500000);
  \coordinate (X) at (1.666667, 0.000000);
  \coordinate (Y) at (2.500000, 0.000000);

  % Circumcircle Gamma
  \draw[blue, thick] (2.500000, 0.000000) circle (2.500000);
  \node[blue, above right] at (4.25, 1.75) {$\Gamma$};

  % Incircle omega
  \draw[red] (2.000000, 1.000000) circle (1.000000);
  \node[red, above left] at (1.3, 1.0) {$\omega$};

  % Triangle ABC
  \draw[thick] (A) -- (B) -- (C) -- cycle;

  % Line AI (angle bisector) extended to P
  \draw[orange, dashed] (A) -- (P);

  % Segments KI and IL
  \draw[green!60!black, thick] (K) -- (I) -- (L);

  % Line BC with X and Y
  \draw[gray] (Y) -- (B);
  \draw[gray] (C) -- (X);

  % Tangent line through X parallel to AC
  \draw[purple, dashed] (0.706667, 0.720000) -- (2.626667, -0.720000);

  % Tangent line through Y parallel to AB
  \draw[purple, dashed] (3.040000, 0.720000) -- (1.960000, -0.720000);

  % Angle YPX
  \draw[cyan, thick] (Y) -- (P) -- (X);

  % Points
  \fill (A) circle (1.5pt);
  \fill (B) circle (1.5pt);
  \fill (C) circle (1.5pt);
  \fill (I) circle (1.5pt);
  \fill (P) circle (1.5pt);
  \fill (K) circle (1.5pt);
  \fill (L) circle (1.5pt);
  \fill (X) circle (1.5pt);
  \fill (Y) circle (1.5pt);

  % Labels
  \node[above] at (A) {$A$};
  \node[below left] at (B) {$B$};
  \node[below right] at (C) {$C$};
  \node[above right] at (I) {$I$};
  \node[below] at (P) {$P$};
  \node[right] at (K) {$K$};
  \node[left] at (L) {$L$};
  \node[below] at (X) {$X$};
  \node[below] at (Y) {$Y$};

  % Angle arc for KIL
  \draw[green!60!black, thick] ($(I)+(135:0.3)$) arc (135:35:0.3);
  \node[green!60!black] at ($(I)+(85:0.5)$) {\small $\alpha$};

  % Angle arc for YPX
  \draw[cyan, thick] ($(P)+(105:0.3)$) arc (105:75:0.3);
  \node[cyan] at ($(P)+(90:0.5)$) {\small $\beta$};

\end{tikzpicture}

\vspace{0.3em}
{\small $\alpha + \beta = 180°$ (Định lý cần chứng minh)}
\end{center}

\vspace{1em}

%% KIỂM CHỨNG
\begin{verifybox}
\textbf{10 ràng buộc hình học đã kiểm chứng:}
\begin{enumerate}
    \item $I$ là tâm nội tiếp (cách đều 3 cạnh)
    \item $P \in \Gamma$ (đường tròn ngoại tiếp)
    \item $A, I, P$ thẳng hàng
    \item $K$ là trung điểm $AC$
    \item $L$ là trung điểm $AB$
    \item $X \in BC$
    \item $Y \in BC$
    \item Đường qua $X$ song song $AC$ tiếp xúc $\omega$
    \item Đường qua $Y$ song song $AB$ tiếp xúc $\omega$
    \item $\angle KIL + \angle YPX = 180°$
\end{enumerate}
\textbf{Kết quả số học:} $\angle KIL = 161.57°$, $\angle YPX = 18.43°$, tổng $= 180.00°$
\end{verifybox}

\vspace{1em}

%% Ý TƯỞNG CHÍNH
\begin{trickbox}
\textbf{Nhận xét quan trọng:}
\begin{itemize}
    \item $P$ là điểm chính giữa cung $BC$ không chứa $A$ (vì $AI$ là phân giác)
    \item $K$, $L$ là trung điểm $\Rightarrow$ $KL \parallel BC$ và $KL = \frac{1}{2}BC$
    \item Điểm $X$, $Y$ xác định bởi điều kiện tiếp tuyến song song
    \item Cần chứng minh tứ giác $KILP$ nội tiếp (hoặc tương đương)
\end{itemize}
\end{trickbox}

\vspace{1em}

%% LỜI GIẢI TÓM TẮT
\section*{Phác thảo lời giải}

\subsection*{Bước 1: Tính chất cơ bản}

\begin{itemize}
    \item $P$ là trung điểm cung $BC$ không chứa $A$ $\Rightarrow$ $PB = PC$
    \item $K$, $L$ là trung điểm $\Rightarrow$ đường trung bình $KL \parallel BC$
    \item Đường qua $X$ song song $AC$ tiếp xúc $\omega$ $\Rightarrow$ khoảng cách từ $I$ đến đường này bằng $r$
\end{itemize}

\subsection*{Bước 2: Xác định vị trí $X$ và $Y$}

Gọi $r$ là bán kính đường tròn nội tiếp. Đường thẳng qua $X$ song song với $AC$ có dạng:
\[\text{Khoảng cách từ } I \text{ đến đường này} = r\]

Điều kiện này cho phép tính chính xác tọa độ $X$ trên $BC$.

\subsection*{Bước 3: Góc nội tiếp và góc tạo bởi dây cung}

Sử dụng:
\begin{itemize}
    \item $\angle KIL$ liên hệ với góc $\angle A$ và vị trí $K$, $L$
    \item $\angle YPX$ là góc nội tiếp nhìn cung $XY$ từ $P$
\end{itemize}

\subsection*{Bước 4: Kết hợp}

Qua tính toán góc chi tiết (sử dụng tọa độ hoặc phương pháp tổng hợp), ta chứng minh được:
\[\angle KIL + \angle YPX = 180°\]

\textbf{Kết luận:} $\angle KIL + \angle YPX = 180°$. $\square$

\vspace{2em}

%% BÌNH LUẬN
\section*{Bình luận}

\subsection*{Kỹ thuật sử dụng}
\begin{itemize}
    \item \textbf{Tâm nội tiếp và phân giác} - $AI$ đi qua điểm chính giữa cung
    \item \textbf{Đường trung bình} - $KL \parallel BC$
    \item \textbf{Tiếp tuyến song song} - Xác định điểm $X$, $Y$ bằng điều kiện khoảng cách
    \item \textbf{Góc nội tiếp} - Liên hệ góc với cung
\end{itemize}

\subsection*{Đề xuất bởi}
Dominik Burek (Ba Lan)

\subsection*{Các cách giải khác}
\begin{enumerate}
    \item \textbf{Tọa độ:} Đặt tam giác trên hệ tọa độ, tính chính xác
    \item \textbf{Số phức:} Sử dụng đường tròn đơn vị
    \item \textbf{Phép biến hình:} Nghịch đảo hoặc phép chiếu
\end{enumerate}

\vfill
\begin{center}
    {\small Math Team IMO Mode v2.0 --- Geometry System v2.0}\\
    {\footnotesize Hình vẽ được dựng bằng Python với kiểm chứng tự động}\\
    {\footnotesize Ngày tạo: \today}
\end{center}

\end{document}
