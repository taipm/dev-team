% Phương pháp điểm rơi trong chứng minh bất đẳng thức
% Tài liệu dành cho Giáo viên và Huấn luyện viên Toán
% Compiled with XeLaTeX for Vietnamese support

\documentclass[11pt,a4paper]{book}

% ===== PACKAGES =====
% Vietnamese support
\usepackage{fontspec}
\usepackage{polyglossia}
\setdefaultlanguage{vietnamese}

% Mathematics
\usepackage{amsmath}
\usepackage{amssymb}
\usepackage{amsthm}
\usepackage{mathtools}

% Graphics
\usepackage{tikz}
\usetikzlibrary{arrows.meta, positioning, calc}

% Layout
\usepackage[margin=2.5cm]{geometry}
\usepackage{fancyhdr}
\usepackage{titlesec}

% Colors
\usepackage{xcolor}
\definecolor{maincolor}{RGB}{0, 100, 150}
\definecolor{accentcolor}{RGB}{200, 50, 50}
\definecolor{notecolor}{RGB}{50, 150, 50}

% Boxes and environments
\usepackage{tcolorbox}
\tcbuselibrary{theorems, skins, breakable}

% Hyperlinks
\usepackage{hyperref}
\hypersetup{
    colorlinks=true,
    linkcolor=maincolor,
    urlcolor=maincolor
}

% ===== THEOREM ENVIRONMENTS =====
\theoremstyle{definition}
\newtheorem{theorem}{Định lý}[chapter]
\newtheorem{lemma}[theorem]{Bổ đề}
\newtheorem{corollary}[theorem]{Hệ quả}
\newtheorem{proposition}[theorem]{Mệnh đề}

\theoremstyle{definition}
\newtheorem{definition}{Định nghĩa}[chapter]
\newtheorem{example}{Ví dụ}[chapter]
\newtheorem{exercise}{Bài tập}[chapter]

\theoremstyle{remark}
\newtheorem*{remark}{Nhận xét}
\newtheorem*{solution}{Lời giải}

% ===== CUSTOM BOXES =====
\newtcolorbox{notebox}{
    colback=notecolor!10,
    colframe=notecolor!80,
    fonttitle=\bfseries,
    title=Ghi chú,
    breakable
}

\newtcolorbox{warningbox}{
    colback=accentcolor!10,
    colframe=accentcolor!80,
    fonttitle=\bfseries,
    title=Cảnh báo,
    breakable
}

\newtcolorbox{keypointsbox}{
    colback=maincolor!10,
    colframe=maincolor!80,
    fonttitle=\bfseries,
    title=Điểm chính,
    breakable
}

% Aliases for LaTeX source
\newenvironment{note}{\begin{notebox}}{\end{notebox}}
\newenvironment{warning}{\begin{warningbox}}{\end{warningbox}}
\newenvironment{keypoints}{\begin{keypointsbox}}{\end{keypointsbox}}

% Algorithm environment for Chapter 2
\newtcolorbox{algorithmbox}[1][]{
    colback=maincolor!5,
    colframe=maincolor!70,
    fonttitle=\bfseries,
    title=Thuật toán,
    #1,
    breakable
}
\newenvironment{algorithm}[1][]{\begin{algorithmbox}[title={#1}]}{\end{algorithmbox}}

% ===== HEADER/FOOTER =====
\pagestyle{fancy}
\fancyhf{}
\fancyhead[LE,RO]{\thepage}
\fancyhead[LO]{\nouppercase{\rightmark}}
\fancyhead[RE]{\nouppercase{\leftmark}}
\renewcommand{\headrulewidth}{0.4pt}

% ===== TITLE FORMATTING =====
\titleformat{\chapter}[display]
    {\normalfont\huge\bfseries\color{maincolor}}
    {\chaptertitlename\ \thechapter}{20pt}{\Huge}

% ===== DOCUMENT INFO =====
\title{%
    \Huge\textbf{Phương pháp điểm rơi}\\[0.5cm]
    \LARGE\textbf{trong chứng minh bất đẳng thức}\\[1cm]
    \large Tài liệu chuyên sâu dành cho Giáo viên và Huấn luyện viên Toán
}
\author{Book Writer Team}
\date{Tháng 1, 2026}

% ===== DOCUMENT =====
\begin{document}

% ===== FRONT MATTER =====
\frontmatter

\maketitle

\tableofcontents

% ===== MAIN MATTER =====
\mainmatter

% ===== CHAPTER 1 =====
\chapter{Tổng quan về bất đẳng thức và điểm rơi}

\begin{flushright}
\textit{``Trong toán học, nghệ thuật đặt câu hỏi có giá trị hơn nghệ thuật giải chúng.''} \\
--- Georg Cantor
\end{flushright}

\section{Giới thiệu}

Bất đẳng thức là một trong những chủ đề quan trọng và thú vị nhất của toán học. Qua các kỳ thi học sinh giỏi quốc gia và quốc tế, bất đẳng thức luôn xuất hiện như một phần không thể thiếu, thử thách khả năng tư duy logic và sáng tạo của học sinh.

Trong chương này, chúng ta sẽ:
\begin{itemize}
    \item Hệ thống hóa các bất đẳng thức cơ bản và điều kiện dấu bằng
    \item Hiểu khái niệm ``điểm rơi'' --- một công cụ chiến lược trong chứng minh bất đẳng thức
    \item Phân loại các dạng điểm rơi thường gặp
    \item Nhận biết vai trò của điểm rơi trong việc chọn phương pháp giải
\end{itemize}

\section{Các bất đẳng thức cơ bản}

Trước khi đi vào phương pháp điểm rơi, chúng ta cần nắm vững các bất đẳng thức nền tảng. Điều quan trọng nhất ở đây là \textbf{điều kiện xảy ra dấu bằng} --- đây chính là ``điểm rơi'' của bất đẳng thức.

\subsection{Bất đẳng thức AM-GM (Cauchy)}

\begin{theorem}[Bất đẳng thức AM-GM]
Cho các số thực không âm $a_1, a_2, \ldots, a_n$. Ta có:
\begin{equation}
\frac{a_1 + a_2 + \cdots + a_n}{n} \geq \sqrt[n]{a_1 a_2 \cdots a_n}
\end{equation}
Dấu bằng xảy ra khi và chỉ khi $a_1 = a_2 = \cdots = a_n$.
\end{theorem}

\begin{note}
Điều kiện dấu bằng ``$a_1 = a_2 = \cdots = a_n$'' cho ta biết \textbf{điểm rơi đối xứng} của bất đẳng thức AM-GM.
\end{note}

\begin{example}
Cho $a, b, c > 0$ với $abc = 1$. Chứng minh rằng:
\[
a + b + c \geq 3
\]

\textbf{Lời giải.} Áp dụng bất đẳng thức AM-GM:
\[
\frac{a + b + c}{3} \geq \sqrt[3]{abc} = \sqrt[3]{1} = 1
\]
Suy ra $a + b + c \geq 3$.

Dấu bằng xảy ra khi $a = b = c$. Kết hợp với $abc = 1$, ta có $a = b = c = 1$.

\textbf{Điểm rơi:} $(a, b, c) = (1, 1, 1)$.
\end{example}

\subsection{Bất đẳng thức Cauchy-Schwarz}

\begin{theorem}[Bất đẳng thức Cauchy-Schwarz]
Cho các số thực $a_1, a_2, \ldots, a_n$ và $b_1, b_2, \ldots, b_n$ với $b_i > 0$. Ta có:
\begin{equation}
\frac{a_1^2}{b_1} + \frac{a_2^2}{b_2} + \cdots + \frac{a_n^2}{b_n} \geq \frac{(a_1 + a_2 + \cdots + a_n)^2}{b_1 + b_2 + \cdots + b_n}
\end{equation}
Dấu bằng xảy ra khi và chỉ khi $\displaystyle\frac{a_1}{b_1} = \frac{a_2}{b_2} = \cdots = \frac{a_n}{b_n}$.
\end{theorem}

\begin{note}
Điều kiện dấu bằng ở đây phức tạp hơn AM-GM: không nhất thiết $a_i = a_j$, mà là các \textbf{tỷ số} $\frac{a_i}{b_i}$ phải bằng nhau.
\end{note}

\begin{example}
Cho $a, b, c > 0$. Chứng minh rằng:
\[
\frac{a^2}{b+c} + \frac{b^2}{c+a} + \frac{c^2}{a+b} \geq \frac{a+b+c}{2}
\]

\textbf{Lời giải.} Áp dụng dạng Engel của Cauchy-Schwarz:
\[
\frac{a^2}{b+c} + \frac{b^2}{c+a} + \frac{c^2}{a+b} \geq \frac{(a+b+c)^2}{2(a+b+c)} = \frac{a+b+c}{2}
\]

Dấu bằng xảy ra khi $\frac{a}{b+c} = \frac{b}{c+a} = \frac{c}{a+b}$, tức là $a = b = c$.

\textbf{Điểm rơi:} $a = b = c$ (giá trị cụ thể tùy thuộc vào ràng buộc).
\end{example}

\subsection{Bất đẳng thức Schur}

\begin{theorem}[Bất đẳng thức Schur]
Cho các số thực không âm $a, b, c$ và $t > 0$. Ta có:
\begin{equation}
a^t(a-b)(a-c) + b^t(b-c)(b-a) + c^t(c-a)(c-b) \geq 0
\end{equation}
Dấu bằng xảy ra khi và chỉ khi:
\begin{itemize}
    \item $a = b = c$, hoặc
    \item Hai trong ba số bằng nhau và số còn lại bằng $0$.
\end{itemize}
\end{theorem}

\begin{note}
Bất đẳng thức Schur có \textbf{hai loại điểm rơi}:
\begin{enumerate}
    \item \textbf{Điểm rơi đối xứng:} $a = b = c$
    \item \textbf{Điểm rơi biên:} $(a, a, 0)$ hoặc hoán vị
\end{enumerate}
Đây là một đặc điểm quan trọng mà chúng ta sẽ gặp lại nhiều lần.
\end{note}

\section{Khái niệm điểm rơi}

\subsection{Định nghĩa}

\begin{definition}[Điểm rơi]
\textbf{Điểm rơi} (equality case) của một bất đẳng thức là tập hợp các giá trị của biến mà tại đó dấu bằng xảy ra.
\end{definition}

Trong bất đẳng thức $f(a, b, c) \geq g(a, b, c)$, điểm rơi là bộ $(a_0, b_0, c_0)$ sao cho:
\[
f(a_0, b_0, c_0) = g(a_0, b_0, c_0)
\]

\subsection{Tại sao điểm rơi quan trọng?}

Điểm rơi đóng vai trò chiến lược trong việc chứng minh bất đẳng thức vì:

\begin{enumerate}
    \item \textbf{Xác nhận tính đúng đắn:} Nếu bất đẳng thức đúng, điểm rơi phải tồn tại và thỏa mãn các ràng buộc.

    \item \textbf{Gợi ý phương pháp:} Biết điểm rơi giúp ta chọn kỹ thuật phù hợp:
    \begin{itemize}
        \item Điểm rơi $a = b = c$ $\Rightarrow$ Thử AM-GM hoặc Schur
        \item Điểm rơi trên biên $\Rightarrow$ Thử kỹ thuật smoothing
        \item Điểm rơi hỗn hợp $\Rightarrow$ Thử phương pháp pqr
    \end{itemize}

    \item \textbf{Kiểm tra lời giải:} Lời giải đúng phải ``bảo toàn'' điểm rơi --- tức là tại điểm rơi, tất cả các đánh giá đều trở thành đẳng thức.
\end{enumerate}

\begin{warning}
Một lỗi phổ biến là đưa ra lời giải không thể đạt dấu bằng tại điểm rơi. Nếu điều này xảy ra, lời giải \textbf{sai hoàn toàn} --- dù các bước biến đổi có vẻ đúng!
\end{warning}

\subsection{Ý nghĩa hình học}

Từ góc độ tối ưu hóa, điểm rơi chính là \textbf{điểm cực trị} của hàm $h(a,b,c) = f(a,b,c) - g(a,b,c)$ trên miền xác định.

\begin{center}
\begin{tikzpicture}[scale=1.2]
    % Axes
    \draw[->] (-0.5,0) -- (4,0) node[right] {$x$};
    \draw[->] (0,-0.5) -- (0,3) node[above] {$f(x)$};

    % Curve
    \draw[thick, blue, domain=0.3:3.5, smooth] plot (\x, {0.5*(\x-2)^2 + 0.5});

    % Minimum point
    \fill[red] (2, 0.5) circle (2pt);
    \node[below] at (2, 0.3) {Điểm rơi};

    % Horizontal line at minimum
    \draw[dashed, gray] (0, 0.5) -- (2, 0.5);

    % Labels
    \node[right, blue] at (3.2, 1.5) {$f(x) \geq c$};
    \node[left] at (0, 0.5) {$c$};
\end{tikzpicture}
\end{center}

\section{Phân loại điểm rơi}

\subsection{Điểm rơi đối xứng}

\begin{definition}
\textbf{Điểm rơi đối xứng} là điểm mà tất cả các biến bằng nhau: $a = b = c$.
\end{definition}

\textbf{Đặc điểm nhận dạng:}
\begin{itemize}
    \item Bất đẳng thức đối xứng hoàn toàn (symmetric)
    \item Ràng buộc đối xứng (ví dụ: $a + b + c = 3$ hoặc $abc = 1$)
    \item Hàm lồi/lõm với gradient triệt tiêu tại $a = b = c$
\end{itemize}

\subsection{Điểm rơi biên}

\begin{definition}
\textbf{Điểm rơi biên} là điểm nằm trên biên của miền xác định, thường có dạng:
\begin{itemize}
    \item $a = 0$ (một biến triệt tiêu)
    \item $a = b, c = 0$ (hai biến bằng nhau, một biến triệt tiêu)
\end{itemize}
\end{definition}

\subsection{Điểm rơi hỗn hợp}

\begin{definition}
\textbf{Điểm rơi hỗn hợp} là điểm mà một số biến bằng nhau, một số khác:
\[
a = b \neq c
\]
\end{definition}

\section{Bài tập}

\subsection{Bài tập cơ bản}

\begin{exercise}[Nhận dạng điểm rơi]
Xác định điểm rơi của các bất đẳng thức sau:
\begin{enumerate}
    \item $a^2 + b^2 \geq 2ab$ với $a, b \in \mathbb{R}$
    \item $\frac{a}{b} + \frac{b}{a} \geq 2$ với $a, b > 0$
    \item $(a + b)(b + c)(c + a) \geq 8abc$ với $a, b, c > 0$
\end{enumerate}
\end{exercise}

\begin{solution}
\begin{enumerate}
    \item $a^2 + b^2 - 2ab = (a-b)^2 \geq 0$. Dấu bằng khi $a = b$.
    \textbf{Điểm rơi:} $a = b$.

    \item Áp dụng AM-GM: $\frac{a}{b} + \frac{b}{a} \geq 2\sqrt{\frac{a}{b} \cdot \frac{b}{a}} = 2$.
    Dấu bằng khi $\frac{a}{b} = \frac{b}{a}$, tức $a = b$.
    \textbf{Điểm rơi:} $a = b$.

    \item Nhân ba BĐT AM-GM: $(a+b)(b+c)(c+a) \geq 8\sqrt{a^2b^2c^2} = 8abc$.
    Dấu bằng khi $a = b$, $b = c$, $c = a$, tức $a = b = c$.
    \textbf{Điểm rơi:} $a = b = c$ (đối xứng).
\end{enumerate}
\end{solution}

\subsection{Bài tập nâng cao}

\begin{exercise}[Điểm rơi biên]
Cho $a, b, c \geq 0$ với $a + b + c = 3$. Chứng minh rằng:
\[
a^3 + b^3 + c^3 \geq 3
\]
Xác định khi nào dấu bằng xảy ra.
\end{exercise}

\begin{solution}
Áp dụng Power Mean:
\[
\sqrt[3]{\frac{a^3 + b^3 + c^3}{3}} \geq \frac{a + b + c}{3} = 1
\]
Suy ra $a^3 + b^3 + c^3 \geq 3$.

Dấu bằng khi $a = b = c = 1$.

\textbf{Điểm rơi:} $(1, 1, 1)$.
\end{solution}

\begin{exercise}[Điểm rơi hỗn hợp - Thách thức]
Cho $a, b, c \geq 0$ với $a + b + c = 2$. Tìm giá trị lớn nhất của:
\[
P = a^2b + b^2c + c^2a
\]
\end{exercise}

\begin{solution}
Thử với $c = 0$, $a + b = 2$: $P = a^2 b = a^2(2-a)$.

Đặt $f(a) = a^2(2-a) = 2a^2 - a^3$, ta có $f'(a) = 4a - 3a^2 = a(4 - 3a)$.

$f'(a) = 0 \Leftrightarrow a = \frac{4}{3}$.

Tại $a = \frac{4}{3}$: $P = \frac{16}{9} \cdot \frac{2}{3} = \frac{32}{27}$.

\textbf{Kết luận:} $P_{\max} = \frac{32}{27}$, đạt tại $\left(\frac{4}{3}, \frac{2}{3}, 0\right)$ và các hoán vị cyclic.

\textbf{Điểm rơi:} Hỗn hợp, nằm trên biên.
\end{solution}

\section{Tóm tắt}

\begin{keypoints}
\begin{enumerate}
    \item Điểm rơi là giá trị biến khi dấu bằng xảy ra
    \item Biết điểm rơi giúp chọn phương pháp phù hợp
    \item Lời giải đúng phải ``bảo toàn'' điểm rơi
    \item Điểm rơi đối xứng: $a = b = c$
    \item Điểm rơi biên: một biến bằng 0 hoặc $(a, a, 0)$
    \item Điểm rơi hỗn hợp: $a = b \neq c$
\end{enumerate}
\end{keypoints}

\section{Tiếp theo}

Trong Chương 2, chúng ta sẽ đi sâu vào \textbf{lý thuyết tổng quát của phương pháp điểm rơi}, bao gồm:
\begin{itemize}
    \item Phương pháp xác định điểm rơi một cách hệ thống
    \item Kỹ thuật Lagrange multipliers
    \item Phương pháp SOS (Sum of Squares)
    \item Phương pháp pqr cho hệ 3 biến
\end{itemize}

% ===== CHAPTER 2 =====
\chapter{Phương pháp điểm rơi - Lý thuyết tổng quát}

\begin{flushright}
\textit{``Một bài toán hay đặt ra không chỉ là thử thách, mà còn là cơ hội để ta hiểu sâu hơn về toán học.''} \\
--- Issai Schur
\end{flushright}

\section{Giới thiệu}

Trong Chương 1, chúng ta đã làm quen với khái niệm điểm rơi và hiểu được tầm quan trọng của nó trong việc chứng minh bất đẳng thức. Câu hỏi tự nhiên đặt ra là: \textit{Làm thế nào để xác định điểm rơi một cách có hệ thống?}

Chương này sẽ trình bày các phương pháp lý thuyết để xác định và sử dụng điểm rơi:
\begin{itemize}
    \item Nguyên lý xác định điểm rơi qua thử điểm đặc biệt và đạo hàm
    \item Phương pháp Lagrange multipliers cho bất đẳng thức có ràng buộc
    \item Phương pháp SOS (Sum of Squares) --- phân tích thành tổng bình phương
    \item Phương pháp pqr cho hệ 3 biến
\end{itemize}

\section{Nguyên lý xác định điểm rơi}

\subsection{Phương pháp thử điểm đặc biệt}

Ý tưởng đơn giản nhất là \textbf{thử các điểm đặc biệt} để tìm điểm rơi. Các điểm cần thử bao gồm:

\begin{enumerate}
    \item \textbf{Điểm đối xứng:} $a = b = c$
    \item \textbf{Điểm biên:} $a = 0$ hoặc $a = b, c = 0$
    \item \textbf{Điểm hỗn hợp:} $a = b \neq c$
\end{enumerate}

\begin{algorithm}[Xác định điểm rơi bằng thử điểm]
Cho bất đẳng thức $f(a,b,c) \geq g(a,b,c)$ với ràng buộc $h(a,b,c) = k$.

\textbf{Bước 1:} Thử $a = b = c$ thỏa mãn ràng buộc. Kiểm tra VT $=$ VP?

\textbf{Bước 2:} Nếu Bước 1 không thành công, thử $a = b, c = 0$.

\textbf{Bước 3:} Nếu Bước 2 không thành công, thử $a = 0, b = c$.

\textbf{Bước 4:} So sánh các giá trị $f - g$ tại các điểm thử để xác định điểm rơi.
\end{algorithm}

\begin{example}[Xác định điểm rơi bằng thử điểm]
Cho $a, b, c \geq 0$ với $a + b + c = 3$. Xác định điểm rơi của:
\[
a^2 + b^2 + c^2 \geq ?
\]

\textbf{Thử điểm:}
\begin{itemize}
    \item $a = b = c = 1$: VT $= 3$
    \item $a = 3, b = c = 0$: VT $= 9$
    \item $a = b = 1.5, c = 0$: VT $= 4.5$
\end{itemize}

\textbf{Kết luận:} Giá trị nhỏ nhất là $3$, đạt tại $a = b = c = 1$.

Vậy bất đẳng thức là $a^2 + b^2 + c^2 \geq 3$ với điểm rơi $(1, 1, 1)$.
\end{example}

\subsection{Phương pháp đạo hàm riêng}

Từ góc độ giải tích, điểm rơi của bất đẳng thức $f(a,b,c) \geq g(a,b,c)$ chính là điểm cực trị của hàm:
\[
h(a,b,c) = f(a,b,c) - g(a,b,c)
\]

Tại điểm cực trị nội, các đạo hàm riêng triệt tiêu:
\[
\frac{\partial h}{\partial a} = \frac{\partial h}{\partial b} = \frac{\partial h}{\partial c} = 0
\]

\begin{theorem}[Điều kiện cần cho cực trị]
Nếu $h(a,b,c)$ có cực trị tại điểm nội $(a_0, b_0, c_0)$ trong miền xác định, thì:
\[
\nabla h(a_0, b_0, c_0) = \mathbf{0}
\]
\end{theorem}

\begin{example}[Sử dụng đạo hàm riêng]
Tìm giá trị nhỏ nhất của $f(a,b,c) = a^2 + b^2 + c^2$ với $a + b + c = 3$, $a,b,c > 0$.

\textbf{Lời giải.} Sử dụng Lagrange:
\[
\mathcal{L}(a,b,c,\lambda) = a^2 + b^2 + c^2 - \lambda(a + b + c - 3)
\]

Đạo hàm riêng:
\begin{align*}
\frac{\partial \mathcal{L}}{\partial a} &= 2a - \lambda = 0 \\
\frac{\partial \mathcal{L}}{\partial b} &= 2b - \lambda = 0 \\
\frac{\partial \mathcal{L}}{\partial c} &= 2c - \lambda = 0
\end{align*}

Từ hệ phương trình, ta có $a = b = c = \frac{\lambda}{2}$.

Với ràng buộc $a + b + c = 3$: $\frac{3\lambda}{2} = 3 \Rightarrow \lambda = 2$.

Vậy $a = b = c = 1$, và $f_{\min} = 3$.
\end{example}

\begin{note}
Phương pháp đạo hàm chỉ cho điểm cực trị nội. Cần kiểm tra thêm điểm biên để có kết luận hoàn chỉnh.
\end{note}

\section{Phương pháp Lagrange Multipliers}

\subsection{Lý thuyết cơ bản}

Phương pháp nhân tử Lagrange là công cụ mạnh để tìm cực trị của hàm với ràng buộc.

\begin{theorem}[Lagrange Multipliers]
Cho $f: \mathbb{R}^n \to \mathbb{R}$ và ràng buộc $g(\mathbf{x}) = c$. Nếu $f$ đạt cực trị tại $\mathbf{x}_0$ trên tập ràng buộc, thì tồn tại $\lambda \in \mathbb{R}$ sao cho:
\[
\nabla f(\mathbf{x}_0) = \lambda \nabla g(\mathbf{x}_0)
\]
\end{theorem}

\textbf{Quy trình:}
\begin{enumerate}
    \item Lập hàm Lagrange: $\mathcal{L}(\mathbf{x}, \lambda) = f(\mathbf{x}) - \lambda(g(\mathbf{x}) - c)$
    \item Giải hệ phương trình: $\nabla \mathcal{L} = \mathbf{0}$
    \item Kiểm tra điểm tìm được là cực đại hay cực tiểu
\end{enumerate}

\subsection{Áp dụng cho bất đẳng thức}

\begin{example}[Lagrange với ràng buộc tích]
Cho $a, b, c > 0$ với $abc = 1$. Chứng minh rằng:
\[
a + b + c \geq 3
\]

\textbf{Lời giải.} Tìm giá trị nhỏ nhất của $f = a + b + c$ với $g = abc = 1$.

Hàm Lagrange:
\[
\mathcal{L} = a + b + c - \lambda(abc - 1)
\]

Đạo hàm:
\begin{align*}
\frac{\partial \mathcal{L}}{\partial a} &= 1 - \lambda bc = 0 \Rightarrow \lambda = \frac{1}{bc} \\
\frac{\partial \mathcal{L}}{\partial b} &= 1 - \lambda ac = 0 \Rightarrow \lambda = \frac{1}{ac} \\
\frac{\partial \mathcal{L}}{\partial c} &= 1 - \lambda ab = 0 \Rightarrow \lambda = \frac{1}{ab}
\end{align*}

Từ $\frac{1}{bc} = \frac{1}{ac} = \frac{1}{ab}$, suy ra $a = b = c$.

Với $abc = 1$: $a^3 = 1 \Rightarrow a = 1$.

Vậy $a = b = c = 1$, và $f_{\min} = 3$.

\textbf{Điểm rơi:} $(1, 1, 1)$.
\end{example}

\begin{example}[Lagrange với bất đẳng thức phức tạp]
Cho $a, b, c > 0$ với $a + b + c = 3$. Tìm giá trị nhỏ nhất của:
\[
P = \frac{1}{a} + \frac{1}{b} + \frac{1}{c}
\]

\textbf{Lời giải.} Hàm Lagrange:
\[
\mathcal{L} = \frac{1}{a} + \frac{1}{b} + \frac{1}{c} - \lambda(a + b + c - 3)
\]

Đạo hàm:
\[
\frac{\partial \mathcal{L}}{\partial a} = -\frac{1}{a^2} - \lambda = 0 \Rightarrow a^2 = -\frac{1}{\lambda}
\]

Tương tự: $b^2 = c^2 = -\frac{1}{\lambda}$.

Do $a, b, c > 0$, ta có $a = b = c$.

Với $a + b + c = 3$: $a = b = c = 1$.

Vậy $P_{\min} = 3$, đạt tại $(1, 1, 1)$.
\end{example}

\subsection{Điều kiện KKT cho bất đẳng thức với ràng buộc bất đẳng thức}

Khi có thêm ràng buộc dạng bất đẳng thức (ví dụ $a \geq 0$), ta sử dụng điều kiện Karush-Kuhn-Tucker (KKT).

\begin{theorem}[Điều kiện KKT]
Cho bài toán tối ưu:
\begin{align*}
\min \quad & f(\mathbf{x}) \\
\text{s.t.} \quad & g_i(\mathbf{x}) \leq 0, \quad i = 1, \ldots, m \\
& h_j(\mathbf{x}) = 0, \quad j = 1, \ldots, p
\end{align*}

Tại điểm cực trị $\mathbf{x}^*$, tồn tại $\mu_i \geq 0$ và $\lambda_j$ sao cho:
\[
\nabla f(\mathbf{x}^*) + \sum_{i=1}^m \mu_i \nabla g_i(\mathbf{x}^*) + \sum_{j=1}^p \lambda_j \nabla h_j(\mathbf{x}^*) = \mathbf{0}
\]
với điều kiện bổ sung: $\mu_i g_i(\mathbf{x}^*) = 0$ (complementary slackness).
\end{theorem}

\begin{note}
Điều kiện complementary slackness $\mu_i g_i(\mathbf{x}^*) = 0$ cho biết: tại điểm tối ưu, ràng buộc hoặc chặt ($g_i = 0$), hoặc nhân tử bằng không ($\mu_i = 0$). Điều này giải thích tại sao điểm rơi thường nằm trên biên!
\end{note}

\section{Phương pháp SOS (Sum of Squares)}

\subsection{Nguyên lý cơ bản}

Phương pháp SOS dựa trên quan sát đơn giản: \textbf{bình phương của số thực luôn không âm}.

\begin{definition}[Phân tích SOS]
Một đa thức $P(x_1, \ldots, x_n)$ được gọi là \textbf{SOS} (Sum of Squares) nếu có thể viết dưới dạng:
\[
P = \sum_{i} Q_i^2
\]
với $Q_i$ là các đa thức.
\end{definition}

\begin{theorem}[SOS implies non-negative]
Nếu $P$ là SOS, thì $P(\mathbf{x}) \geq 0$ với mọi $\mathbf{x} \in \mathbb{R}^n$.
\end{theorem}

\subsection{Dạng SOS chuẩn cho 3 biến đối xứng}

Với bất đẳng thức đối xứng 3 biến, dạng SOS phổ biến là:
\begin{equation}
(a-b)^2 S_c + (b-c)^2 S_a + (c-a)^2 S_b \geq 0
\end{equation}
trong đó $S_a, S_b, S_c$ là các hệ số (có thể phụ thuộc vào $a, b, c$).

\begin{theorem}[Điều kiện đủ cho SOS đối xứng]
Bất đẳng thức $(a-b)^2 S_c + (b-c)^2 S_a + (c-a)^2 S_b \geq 0$ đúng nếu:
\begin{itemize}
    \item $S_a, S_b, S_c \geq 0$, hoặc
    \item $S_a + S_b + S_c \geq 0$ và $S_a S_b + S_b S_c + S_c S_a \geq 0$
\end{itemize}
\end{theorem}

\subsection{Ví dụ phương pháp SOS}

\begin{example}[SOS cơ bản]
Chứng minh rằng với $a, b, c \geq 0$:
\[
a^3 + b^3 + c^3 \geq 3abc
\]

\textbf{Lời giải.} Biến đổi:
\begin{align*}
a^3 + b^3 + c^3 - 3abc &= (a + b + c)(a^2 + b^2 + c^2 - ab - bc - ca) \\
&= \frac{1}{2}(a + b + c)[(a-b)^2 + (b-c)^2 + (c-a)^2]
\end{align*}

Đây là SOS với $S_a = S_b = S_c = \frac{1}{2}(a + b + c) \geq 0$.

\textbf{Điểm rơi:}
\begin{itemize}
    \item $a = b = c$ (các bình phương đều bằng 0), hoặc
    \item $a + b + c = 0$ (với $a, b, c \geq 0$ thì $a = b = c = 0$)
\end{itemize}
\end{example}

\begin{example}[SOS nâng cao - IMO 1983/6]
Cho $a, b, c$ là các cạnh của một tam giác. Chứng minh rằng:
\[
a^2b(a-b) + b^2c(b-c) + c^2a(c-a) \geq 0
\]

\textbf{Lời giải.} Ta phân tích biểu thức dưới dạng SOS. Đặt:
\[
P = a^2b(a-b) + b^2c(b-c) + c^2a(c-a)
\]

Sau khi biến đổi (khá phức tạp), ta được:
\[
P = \frac{1}{2}[(a-b)^2(a+b-c)c + (b-c)^2(b+c-a)a + (c-a)^2(c+a-b)b]
\]

Do $a, b, c$ là các cạnh tam giác, ta có:
\begin{itemize}
    \item $a + b > c \Rightarrow a + b - c > 0$
    \item $b + c > a \Rightarrow b + c - a > 0$
    \item $c + a > b \Rightarrow c + a - b > 0$
\end{itemize}

Vậy $P \geq 0$.

\textbf{Điểm rơi:} $a = b$ hoặc $b = c$ hoặc $c = a$ (tam giác cân).
\end{example}

\subsection{Kỹ thuật tìm hệ số SOS}

\begin{algorithm}[Tìm phân tích SOS]
Cho đa thức đối xứng $P(a,b,c)$ bậc $n$.

\textbf{Bước 1:} Viết $P$ dưới dạng $(a-b)^2 S_c + (b-c)^2 S_a + (c-a)^2 S_b + R$

\textbf{Bước 2:} Xác định $S_a, S_b, S_c$ sao cho $R = 0$ hoặc $R$ cũng là SOS.

\textbf{Bước 3:} Kiểm tra $S_a, S_b, S_c \geq 0$ với mọi $a, b, c$ trong miền xác định.

\textbf{Gợi ý:} Thử $S_c = \alpha a + \beta b + \gamma c$ với các hệ số cần tìm.
\end{algorithm}

\section{Phương pháp pqr}

\subsection{Giới thiệu}

Phương pháp pqr là kỹ thuật mạnh cho bất đẳng thức 3 biến, đặc biệt hiệu quả khi điểm rơi không đối xứng hoàn toàn.

\begin{definition}[Đại lượng pqr]
Cho $a, b, c$ là ba số thực. Đặt:
\begin{align*}
p &= a + b + c & \text{(tổng)} \\
q &= ab + bc + ca & \text{(tổng các tích đôi một)} \\
r &= abc & \text{(tích)}
\end{align*}
\end{definition}

\begin{theorem}[Vieta ngược]
Các số $a, b, c$ là ba nghiệm của phương trình:
\begin{equation}
x^3 - px^2 + qx - r = 0
\end{equation}
\end{theorem}

\subsection{Biểu diễn đa thức theo pqr}

Mọi đa thức đối xứng của $a, b, c$ đều có thể biểu diễn qua $p, q, r$:

\begin{align*}
a^2 + b^2 + c^2 &= p^2 - 2q \\
a^3 + b^3 + c^3 &= p^3 - 3pq + 3r \\
a^2b + ab^2 + b^2c + bc^2 + c^2a + ca^2 &= pq - 3r \\
a^2bc + ab^2c + abc^2 &= pr \\
(a-b)^2(b-c)^2(c-a)^2 &= p^2q^2 - 4p^3r - 4q^3 + 18pqr - 27r^2
\end{align*}

\subsection{Điều kiện thực}

\begin{theorem}[Điều kiện để $a, b, c$ thực]
Ba số $a, b, c$ (nghiệm của $x^3 - px^2 + qx - r = 0$) là thực khi và chỉ khi:
\begin{equation}
T(p,q,r) = p^2q^2 - 4p^3r - 4q^3 + 18pqr - 27r^2 \geq 0
\end{equation}

Lưu ý: $T(p,q,r) = (a-b)^2(b-c)^2(c-a)^2$
\end{theorem}

\subsection{Định lý chính của pqr}

\begin{theorem}[Tejs' Theorem]
Khi hai trong ba đại lượng $p, q, r$ cố định, đại lượng thứ ba đạt cực đại hoặc cực tiểu khi:
\begin{itemize}
    \item Hai trong ba biến $a, b, c$ bằng nhau, hoặc
    \item Một trong ba biến $a, b, c$ bằng $0$
\end{itemize}
\end{theorem}

\begin{corollary}[Điểm rơi của đa thức đối xứng]
Mọi đa thức đối xứng bậc $\leq 5$ với các biến thực không âm $a, b, c$ có cực trị toàn cục sẽ đạt giá trị đó tại các bộ $(a, b, c)$ với:
\begin{enumerate}
    \item Hai biến bằng nhau: $(t, t, s)$, hoặc
    \item Một biến bằng 0: $(a, b, 0)$
\end{enumerate}
\end{corollary}

\subsection{Ví dụ phương pháp pqr}

\begin{example}[pqr cơ bản]
Cho $a, b, c \geq 0$ với $a + b + c = 3$. Chứng minh rằng:
\[
ab + bc + ca \leq 3
\]

\textbf{Lời giải.} Với $p = 3$ cố định, ta cần tìm $q_{\max}$.

Theo Tejs' Theorem, cực trị đạt khi $a = b$ hoặc một biến bằng 0.

\textbf{Trường hợp 1:} $a = b = t$, $c = 3 - 2t$ với $t \in [0, 1.5]$
\[
q = t^2 + 2t(3-2t) = t^2 + 6t - 4t^2 = -3t^2 + 6t
\]
$q'(t) = -6t + 6 = 0 \Rightarrow t = 1$

Tại $t = 1$: $a = b = 1, c = 1$, $q = 3$.

\textbf{Trường hợp 2:} $c = 0$, $a + b = 3$
\[
q = ab \leq \left(\frac{a+b}{2}\right)^2 = \frac{9}{4} < 3
\]

Vậy $q_{\max} = 3$, đạt tại $a = b = c = 1$.
\end{example}

\begin{example}[pqr nâng cao]
Cho $a, b, c \geq 0$ với $a + b + c = 2$. Tìm giá trị lớn nhất của:
\[
P = a^2b + b^2c + c^2a
\]

\textbf{Lời giải.} Đây là bất đẳng thức cyclic, không đối xứng.

Theo Tejs' Theorem, thử các trường hợp biên:

\textbf{Trường hợp 1:} $a = b = t$, $c = 2 - 2t$
\[
P = t^2 \cdot t + t^2(2-2t) + (2-2t)^2 t = t^3 + 2t^2 - 2t^3 + 4t(1-t)^2
\]
Đơn giản hóa và tìm cực trị...

\textbf{Trường hợp 2:} $c = 0$, $a + b = 2$
\[
P = a^2b = a^2(2-a)
\]
$P'(a) = 4a - 3a^2 = a(4 - 3a) = 0 \Rightarrow a = \frac{4}{3}$

Tại $a = \frac{4}{3}$, $b = \frac{2}{3}$:
\[
P = \frac{16}{9} \cdot \frac{2}{3} = \frac{32}{27}
\]

So sánh các giá trị, ta có $P_{\max} = \frac{32}{27}$.

\textbf{Điểm rơi:} $\left(\frac{4}{3}, \frac{2}{3}, 0\right)$ và các hoán vị cyclic.
\end{example}

\subsection{Miền $(p, q, r)$ khả thi}

\begin{definition}[Miền Schur]
Với $p$ cố định và $a, b, c \geq 0$, miền các cặp $(q, r)$ khả thi được giới hạn bởi:
\begin{itemize}
    \item Đường cong $T(p, q, r) = 0$ (điều kiện thực)
    \item Ràng buộc $r \geq 0$ (tích không âm)
    \item Ràng buộc $q \leq \frac{p^2}{3}$ (từ AM-GM cho $a^2 + b^2 + c^2$)
\end{itemize}
\end{definition}

\section{Bài tập}

\subsection{Bài tập cơ bản}

\begin{exercise}[Lagrange cơ bản]
Cho $a, b, c > 0$ với $a + b + c = 6$. Tìm giá trị lớn nhất của $abc$.
\end{exercise}

\begin{solution}
Hàm Lagrange: $\mathcal{L} = abc - \lambda(a + b + c - 6)$

Đạo hàm:
\[
\frac{\partial \mathcal{L}}{\partial a} = bc - \lambda = 0 \Rightarrow \lambda = bc
\]

Tương tự: $\lambda = ac = ab$. Suy ra $a = b = c = 2$.

$abc_{\max} = 8$, đạt tại $(2, 2, 2)$.
\end{solution}

\begin{exercise}[SOS đơn giản]
Sử dụng SOS, chứng minh: $(a + b)(b + c)(c + a) \geq 8abc$ với $a, b, c > 0$.
\end{exercise}

\begin{solution}
Áp dụng AM-GM cho từng cặp:
\[
a + b \geq 2\sqrt{ab}, \quad b + c \geq 2\sqrt{bc}, \quad c + a \geq 2\sqrt{ca}
\]

Nhân ba bất đẳng thức:
\[
(a+b)(b+c)(c+a) \geq 8\sqrt{a^2b^2c^2} = 8abc
\]

Hoặc dùng SOS:
\[
(a+b)(b+c)(c+a) - 8abc = a(b-c)^2 + b(c-a)^2 + c(a-b)^2 \geq 0
\]

\textbf{Điểm rơi:} $a = b = c$.
\end{solution}

\subsection{Bài tập nâng cao}

\begin{exercise}[pqr nâng cao]
Cho $a, b, c \geq 0$ với $a + b + c = 1$. Chứng minh rằng:
\[
a^2 + b^2 + c^2 + 3abc \geq \frac{1}{3}
\]
\end{exercise}

\begin{solution}
Với $p = 1$, ta có:
\[
a^2 + b^2 + c^2 = p^2 - 2q = 1 - 2q
\]

Cần chứng minh: $1 - 2q + 3r \geq \frac{1}{3}$, tức là $3r \geq 2q - \frac{2}{3}$.

Theo Tejs' Theorem, thử $a = b = t$, $c = 1 - 2t$:
\begin{align*}
q &= t^2 + 2t(1-2t) = t^2 + 2t - 4t^2 = -3t^2 + 2t \\
r &= t^2(1-2t)
\end{align*}

VT $- \frac{1}{3} = 2t^2 + (1-2t)^2 + 3t^2(1-2t) - \frac{1}{3}$

Sau khi đơn giản hóa, ta thấy VT $\geq \frac{1}{3}$.

\textbf{Điểm rơi:} $a = b = c = \frac{1}{3}$.
\end{solution}

\begin{exercise}[Lagrange với hai ràng buộc]
Cho $a, b, c > 0$ với $a + b + c = 3$ và $ab + bc + ca = 3$. Chứng minh $a = b = c = 1$.
\end{exercise}

\begin{solution}
Từ $p = 3$ và $q = 3$, ta có:
\[
a^2 + b^2 + c^2 = p^2 - 2q = 9 - 6 = 3
\]

Mặt khác, bất đẳng thức Cauchy-Schwarz:
\[
(a^2 + b^2 + c^2)(1 + 1 + 1) \geq (a + b + c)^2
\]
\[
3 \cdot 3 \geq 9 \quad \checkmark
\]

Dấu bằng khi $a = b = c$. Kết hợp $a + b + c = 3$, ta có $a = b = c = 1$.
\end{solution}

\section{Tóm tắt}

\begin{keypoints}
\begin{enumerate}
    \item \textbf{Phương pháp thử điểm:} Thử các điểm đặc biệt $(a,a,a)$, $(a,a,0)$, $(a,b,0)$
    \item \textbf{Đạo hàm riêng:} Tại điểm cực trị nội, $\nabla h = 0$
    \item \textbf{Lagrange multipliers:} $\nabla f = \lambda \nabla g$ tại điểm tối ưu
    \item \textbf{Điều kiện KKT:} Complementary slackness giải thích điểm rơi trên biên
    \item \textbf{Phương pháp SOS:} Viết $P = \sum (a-b)^2 S_c \geq 0$
    \item \textbf{Phương pháp pqr:} $p = a+b+c$, $q = ab+bc+ca$, $r = abc$
    \item \textbf{Tejs' Theorem:} Cực trị đạt khi hai biến bằng nhau hoặc một biến bằng 0
\end{enumerate}
\end{keypoints}

\section{Tiếp theo}

Trong Chương 3, chúng ta sẽ áp dụng các lý thuyết này vào \textbf{điểm rơi đối xứng} --- trường hợp quan trọng nhất khi $a = b = c$. Chúng ta sẽ học:
\begin{itemize}
    \item Nhận dạng bất đẳng thức có điểm rơi đối xứng
    \item Kỹ thuật chuẩn hóa (normalization)
    \item Kỹ thuật Schur và các biến thể
    \item Phương pháp mixing variables
\end{itemize}

% ===== PLACEHOLDER FOR FUTURE CHAPTERS =====

\chapter{Điểm rơi đối xứng}
\textit{(Nội dung sẽ được bổ sung...)}

\chapter{Điểm rơi biên}
\textit{(Nội dung sẽ được bổ sung...)}

\chapter{Điểm rơi hỗn hợp và các trường hợp đặc biệt}
\textit{(Nội dung sẽ được bổ sung...)}

\chapter{Bất đẳng thức trong các kỳ thi Olympic}
\textit{(Nội dung sẽ được bổ sung...)}

\chapter{Bài tập tổng hợp và phương pháp giảng dạy}
\textit{(Nội dung sẽ được bổ sung...)}

% ===== BACK MATTER =====
\backmatter

\appendix
\chapter{Các bất đẳng thức tham chiếu}
\textit{(Nội dung sẽ được bổ sung...)}

\chapter{Công cụ tính toán}
\textit{(Nội dung sẽ được bổ sung...)}

\chapter{Đáp án bài tập tự luyện}
\textit{(Nội dung sẽ được bổ sung...)}

\end{document}
